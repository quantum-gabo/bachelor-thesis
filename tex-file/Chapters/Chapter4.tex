%----------------------------------------------------------------------------------------

% Chapter 4

\chapter{\texorpdfstring{Results $\&$ Discussion}{Results \& Discussion}}  % Main chapter title
The magnetic monolayers under investigation belong to the trigonal system within the hexagonal family, specifically space group $P\bar{3}$ (No. 147). Prior to performing electronic structure calculations using the PBE functional, a vacuum spacing of approximately 21.8 \AA \ was introduced to the two-dimensional system to prevent interactions between adjacent layers, thereby minimizing van der Waals interactions.

To ensure computational efficiency and accuracy, we first determined the appropriate energy cut-off for the plane wave basis set for both the ferromagnetic (FM) and antiferromagnetic (AFM) phases of each monolayer. A Monkhorst-Pack $9 \times 9 \times 1$ k-point grid was initially employed. Using a convergence criterion of less than 1 meV per formula unit (f.u.), an energy cut-off of 500 eV was selected. Additionally, $\Gamma$-centered k-point sampling was employed to preserve the hexagonal symmetry.
\begin{figure}[H]
	\centering
	\includegraphics[width=\linewidth]{Figures/crystal.png},
	\caption{Top and side view of XGeTe$_{3} (X=Cr, Mn, Fe)$ monolayer}
	\label{fig:4.1}
\end{figure}

Following the determination of the energy cut-off, we optimized the k-point grid necessary for accurate electronic structure calculations. The lattice parameters were computed by fitting the two-dimensional equation of state, as described earlier. A convergence criterion of 1 mÅ was applied to obtain an optimal k-point grid of $10 \times 10 \times 1$ for subsequent calculations.

\label{Chapter4} % For referencing the chapter elsewhere, use \ref{Chapter1} 

\lhead{Chapter 4. \emph{Results $\&$ Discussion}} % This is for the header on each page - perhaps a shortened title

\begin{figure}[H]  
	\centering  
	\begin{subfigure}{.49\textwidth}  
		\centering  
		\begin{picture}(0,0)  
			\put(-12,130){\textbf{(a)}}  
		\end{picture}  
		\includegraphics[width=.95\linewidth]{Figures/plotCrcompENCUT.png}  
	\end{subfigure}%  
	\hfill% Fill space between subfigures  
	\begin{subfigure}{.49\textwidth}  
		\centering  
		\begin{picture}(0,0)  
			\put(-14,130){\textbf{(d)}}  
		\end{picture}  
		\includegraphics[width=.95\linewidth]{Figures/plotkpCreqs.png}  
	\end{subfigure}\\
	
	\begin{subfigure}{.49\textwidth}  
		\centering  
		\begin{picture}(0,0)  
			\put(-12,130){\textbf{(b)}}  
		\end{picture}  
		\includegraphics[width=.95\linewidth]{Figures/plotMncompENCUT.png}  
	\end{subfigure}%  
	\hfill % Fill space between subfigures  
	\begin{subfigure}{.49\textwidth}  
		\centering  
		\begin{picture}(0,0)  
			\put(-14,130){\textbf{(e)}}  
		\end{picture}  
		\includegraphics[width=.95\linewidth]{Figures/plotkpMneqs.png}  
	\end{subfigure}\\
	
	\begin{subfigure}{.49\textwidth}  
		\centering  
		\begin{picture}(0,0)  
			\put(-119,2){\textbf{(c)}}  
		\end{picture}  
		\includegraphics[width=\linewidth]{Figures/plotFecompENCUT.png}  
	\end{subfigure}%  
	\hfill % Fill space between subfigures  
	\begin{subfigure}{.49\textwidth}  
		\centering  
		\begin{picture}(0,0)  
			\put(-119,2){\textbf{(f)}}  
		\end{picture}  
		\includegraphics[width=\linewidth]{Figures/plotkpFeeqs.png}  
	\end{subfigure}  
	\caption{Energy convergence plots for each monolayer. Panels $(\mathbf{a}), (\mathbf{b}), (\mathbf{c})$, with the x-axis representing the cut-off energy, show the convergence of energy with respect to the plane-wave cut-off for both the FM and AFM phases. The average total energy per formula unit (f.u.) converges at 400 eV, with 500 eV selected for further calculations to ensure accurate electronic property description. Panels $(\mathbf{d}), (\mathbf{e}), (\mathbf{f})$, where the x-axis represents the k-point grid $k \times k \times 1$, illustrate that convergence is reached at $8 \times 8 \times 1$. A final grid of $10 \times 10 \times 1$ was used for subsequent calculations.}  
	\label{fig:4.2}  
\end{figure}  

\section{CrGeTe\texorpdfstring{$_3$}{} monolayer}

\subsection{Electronic properties using PBE functional}

With the cut-off energy and k-point grid established, we proceeded to determine the lattice parameters by fitting the total energy to a two-dimensional equation of state, as shown in Fig. \ref{fig:4.3}. We recorded the total magnetic moment, the magnetic moments of the two Cr atoms, the total energy of the system, and the fractional coordinates. These calculations were performed for both ferromagnetic (FM) and antiferromagnetic (AFM) phases.

\begin{figure}[H]
	\begin{minipage}[b]{.45\linewidth}
		\centering
		\includegraphics[width=0.9\linewidth]{Figures/eqs_cgt.png}
		\vspace{-5cm}
		\captionof{figure}{Left panel: Two-dimensional equation of state for the CrGeTe\(_3\) monolayer in FM and AFM phases. The FM phase is more stable, as indicated by $\Delta E = E_{AFM} - E_{FM} = 0.04$ eV/f.u. Right panel: Electronic and magnetic properties obtained using the PBE functional for both phases, confirming semiconductor behavior in line with theoretical predictions \supercite{Gong2017, He2019, Zhang2015}.}
        \label{fig:4.3}
	\end{minipage}\hfill
	\begin{minipage}[b]{.55\linewidth}
		\centering
		\resizebox{8cm}{!}{
	\begin{tabular}{ccccccc}
	\toprule
	\toprule
	Properties          & \multicolumn{3}{c}{FM phase} & \multicolumn{3}{c}{AFM phase} \\ 
	\midrule
	Space group         & \multicolumn{3}{c}{P-3}      & \multicolumn{3}{c}{P-3}       \\
	$a=b$ (\AA)         & \multicolumn{3}{c}{6.91}     & \multicolumn{3}{c}{6.86}      \\
	$c$  (\AA)          & \multicolumn{3}{c}{21.82}    & \multicolumn{3}{c}{21.82}     \\
	$\gamma$(°)         & \multicolumn{3}{c}{120}      & \multicolumn{3}{c}{120}       \\
	Area (\AA)          & \multicolumn{3}{c}{41.37}    & \multicolumn{3}{c}{40.76}     \\
	Total Energy ($eV$) & \multicolumn{3}{c}{-48.362}   & \multicolumn{3}{c}{-48.284}    \\
	\midrule
	\multirow{2}{*}{$\mu_{B}$}       & Cr (1)    & Cr (2)       & Tot   &   Cr (1)     & Cr(2)       & Tot \\
	&  3.15     & 3.15         & 5.81  &   3.07       & -3.07       &  0.00  \\ 
	\midrule
	\multirow{2}{*}{Band gap ($eV$)} & $\uparrow$& $\downarrow$ & Tot   &   $\uparrow$ &$\downarrow$ & Tot \\
	&  0.51     & 0.39         & 0.37  &   0.46       &  0.46       &  0.46  \\ 
	\midrule                                                              
	sites   & u      & v      & w      &   u    &   v     &   w    \\
	Cr(1)   & 0.6666 & 0.3333 & 0.5000 & 0.6666 &  0.3333 & 0.5000 \\
	Cr(2)   & 0.3333 & 0.6666 & 0.5000 & 0.3333 &  0.6666 & 0.5000 \\
	Ge(1)   & 0.0000 & 0.0000 & 0.5555 & 0.0000 &  0.0000 & 0.5555 \\ 
	Ge(2)   & 0.0000 & 0.0000 & 0.4445 & 0.0000 &  0.0000 & 0.4445 \\
	Te(1)   & 0.0002 & 0.3740 & 0.4211 & 0.0001 &  0.3746 & 0.4195 \\
	Te(2)   & 0.9998 & 0.6259 & 0.5789 & 0.9999 &  0.6254 & 0.5805 \\
	Te(3)   & 0.6259 & 0.6262 & 0.4211 & 0.6254 &  0.6255 & 0.4195 \\
	Te(4)   & 0.3740 & 0.3738 & 0.5789 & 0.3746 &  0.3745 & 0.5805 \\
	Te(5)   & 0.3738 & 0.9998 & 0.4211 & 0.3745 &  0.9999 & 0.4195 \\
	Te(6)   & 0.6262 & 0.0002 & 0.5789 & 0.6255 &  0.0000 & 0.5805 \\
	\bottomrule
	\bottomrule
\end{tabular}
}
	\end{minipage}
\end{figure}
	
After optimizing the lattice parameters, we conducted self-consistent calculations to examine the density of states (DOS) and the band structure, as illustrated in Fig. \ref{fig:4.4}. In the ferromagnetic CrGeTe\(_3\) monolayer, the valence band is primarily composed of Te p-orbitals, spanning the energy range from -4.61 eV to 4.12 eV. Significant hybridization between the Cr $t_{2g}$ orbitals ($d_{xy}$, $d_{yz}$, and $d_{zx}$) and the Te p-orbitals is observed in the spin-down channel across the entire energy range, indicating that the PBE functional does not fully capture the localized nature of the Cr $t_{2g}$ states.

\begin{figure}[H]
	\begin{subfigure}{.5\textwidth}
		\centering
		\includegraphics[width=1\linewidth]{Figures/bandsplot_cgt_fm_pbe.png}
	\end{subfigure}%
	\begin{minipage}{.5\textwidth}
		\vspace{-12.6cm}
		\centering
		\includegraphics[width=1\linewidth]{Figures/dosplot_cgt_fm_pbe.png}
		\captionsetup{justification=centering}
	\end{minipage}
	\caption{Band structure (left) and density of states (DOS) (right) for the ferromagnetic CrGeTe$_{3}$ monolayer calculated using the PBE functional. The red and blue colors in the band structure plot represent the spin-up and spin-down channels, respectively. In the DOS plot, the red arrow denotes the spin-up channel, and the blue dashed arrow represents the spin-down channel. The Fermi level is indicated by the red line in both plots.}
	\label{fig:4.4}
\end{figure}

\subsection{Electronic properties using PBESol functional}
\label{subsection.cgt_pbesol}

To enhance the accuracy of our calculations, we employed the PBESol functional, which, to our knowledge, has not been previously applied to this monolayer in theoretical studies. PBESol was chosen for its reduced gradient density dependence, allowing for a more accurate determination of the lattice parameter.


\begin{figure}[H]
	\begin{minipage}[b]{.45\linewidth}
		\centering
		\includegraphics[width=0.9\linewidth]{Figures/eqs_cgtpbesol.png}
		\vspace{-5cm}
		\captionof{figure}{Left panel: Two-dimensional equation of state for the CrGeTe\(_3\) monolayer in the FM and AFM phases. The FM phase is more stable, as indicated by $\Delta E = E_{\text{AFM}} - E_{\text{FM}} = 0.033$ eV/atom. Right panel: Electronic and magnetic properties calculated with the PBESol functional. Compared to the PBE functional (Table \ref{fig:4.3}), PBESol shows a slight underestimation of electronic and magnetic properties. However, using the experimental lattice parameter of $6.82$ \AA \supercite{Gong2017}, PBESol demonstrates a smaller error (0.73\%) compared to PBE (0.17\%).}. 
        \label{fig:4.5}
	\end{minipage}\hfill
	\begin{minipage}[b]{.55\linewidth}
		\centering
		\resizebox{8cm}{!}{
\begin{tabular}{ccccccc}
	\toprule
	\toprule
	Properties          & \multicolumn{3}{c}{FM phase} & \multicolumn{3}{c}{AFM phase} \\
	\midrule
	Space group         & \multicolumn{3}{c}{P-3}      & \multicolumn{3}{c}{P-3}       \\
	$a=b$ (\AA)         & \multicolumn{3}{c}{6.79}     & \multicolumn{3}{c}{6.74}      \\
	$c$  (\AA)          & \multicolumn{3}{c}{21.82}    & \multicolumn{3}{c}{21.82}     \\
	$\gamma$(°)         & \multicolumn{3}{c}{120}      & \multicolumn{3}{c}{120}       \\
	Area (\AA)          & \multicolumn{3}{c}{39.87}    & \multicolumn{3}{c}{39.31}     \\
	Total Energy ($eV$) & \multicolumn{3}{c}{-52.176}   & \multicolumn{3}{c}{-52.110}    \\
	\midrule
	\multirow{2}{*}{$\mu_{B}$}        & Cr (1)    & Cr (2)       & Tot   &   Cr (1)     & Cr(2)       & Tot    \\
	&  3.08     & 3.08         & 5.75  &   2.98       &  -2.98      &  0.00     \\ 
	\midrule
	\multirow{2}{*}{Band gap ($eV$)}  & $\uparrow$& $\downarrow$ & Tot   &   $\uparrow$ &$\downarrow$ & Tot    \\
	&  0.49     & 0.18         &  0.18 &   0.32       &  0.32       &  0.32 \\ 
	\midrule                                                              
	sites  & u      & v      & w       &   u    &    v    &     w    \\
	Cr(1)  & 0.6666 & 0.3333 & 0.5000  & 0.6666 &  0.3333 & 0.5000   \\
	Cr(2)  & 0.3333 & 0.6666 & 0.5000  & 0.3333 &  0.6666 & 0.5000   \\
	Ge(1)  & 0.0000 & 0.0000 & 0.5551  & 0.0000 &  0.0000 & 0.5550   \\ 
	Ge(2)  & 0.0000 & 0.0000 & 0.4449  & 0.0000 &  0.0000 & 0.4449   \\
	Te(1)  & 0.0001 & 0.3766 & 0.4221  & 0.0001 &  0.3775 & 0.4203   \\
	Te(2)  & 0.9999 & 0.6234 & 0.5779  & 0.9999 &  0.6225 & 0.5797   \\
	Te(3)  & 0.6234 & 0.6235 & 0.4211  & 0.6225 &  0.6226 & 0.4203   \\
	Te(4)  & 0.3766 & 0.3765 & 0.5779  & 0.3775 &  0.3774 & 0.5797   \\
	Te(5)  & 0.3765 & 0.9999 & 0.4221  & 0.3774 &  0.9999 & 0.4203   \\
	Te(6)  & 0.6235 & 0.0001 & 0.5779  & 0.6226 &  0.0001 & 0.5797   \\
	\bottomrule
	\bottomrule
\end{tabular}
		}
	\end{minipage}
\end{figure}

Next, we calculated the band structures and density of states (DOS) using the PBESol functional, as shown in Fig. \ref{fig:4.6}. Compared to the PBE functional, the band gap is significantly underestimated with the PBESol functional. Additionally, the $e_g$ states (comprising $d_{z^2}$ and $d_{x^2-y^2}$ orbitals) are localized across the energy range. However, as observed in Fig. \ref{fig:4.4}, the $t_{2g}$ states exhibit hybridization with Te $p$ orbitals in the spin-up channel.

\begin{figure}[H]
	\begin{subfigure}{.5\textwidth}
		\centering
		\includegraphics[width=1\linewidth]{Figures/bandsplot_cgt_fm_pbesol.png}		
	\end{subfigure}%
	\begin{minipage}{.5\textwidth}
		\vspace{-12.4cm}
		\centering
		\includegraphics[width=1\linewidth]{Figures/dosplot_cgt_fm_pbesol.png}
		\captionsetup{justification=centering}
	\end{minipage}
	\caption{Band structure (left) and density of states (DOS) plot (right) for the ferromagnetic CrGeTe$_{3}$ monolayer calculated using the PBESol functional. The red and blue curves correspond to the spin-up and spin-down channels, respectively. The Fermi level is indicated by a red line in both plots. The underestimation of the band gap compared to the PBE functional is evident.}
\label{fig:4.6}
\end{figure}




\subsection{Magnetic and electronic properties using PBE and PBESol functionals with  Hubbard U corrections}

To enhance the accuracy of our computational studies on this monolayer, we incorporated Hubbard $U$ corrections using the Dudarev approach \supercite{Dudarev1998}. This correction is essential for adjusting the magnetic moment and band gap of CrGeTe$_3$ in its ferromagnetic stable phase. Given the experimental lattice parameters for this two-dimensional system are $a=b=6.82\, \text{\AA}$ \supercite{Gong2017}, and the theoretical band gap value is $0.91\, \text{eV}$ \supercite{Wang2019} obtained using the hybrid functional HSE06 \supercite{Krukau2006}, we employed two functionals: PBE and PBESol, in conjunction with the DFT$+$U formalism.

To determine an appropriate value for $U_{\text{eff}} = U - J$, we set the exchange interaction $J$ to zero and explored a range of on-site Coulomb repulsion potentials $U = U_{\text{eff}}$ from 0.0 to 3.0 eV using the PBE functional. For the PBESol functional, the range for $U$ extended from 0.0 to 4.0 eV. These ranges were chosen to ensure the preservation of the structure with Hubbard $U$ corrections. Notably, the space group corresponds to number 147, and this preservation of crystal symmetry is observed for both the ferromagnetic (FM) and antiferromagnetic (AFM) phases.

\begin{table}[H]
	\centering
	\sisetup{table-format=3.0, table-number-alignment=center, group-separator={,}}
	\setlength{\extrarowheight}{0.5ex}
	\caption{Calculated electronic and magnetic properties for FM and AFM phases of CrGeTe$_3$ monolayer using the PBE functional. Both magnetic phases exhibit semiconductor behavior and maintain symmetry conservation (space group 147), similar to the CGT monolayer.}
	\begin{tabular}{cccccc|ccc}
		\toprule
		\toprule
		\rowcolor{WhiteSmoke!70!Lavender}
		Magnetic phase & \multicolumn{8}{c}{Ferromagnetic} \\
		\midrule
		\multirow{2}{*}{Properties} & \multirow{2}{*}{$a=b$(\AA)}& \multirow{2}{*}{Total E(eV)} & \multicolumn{3}{c}{$\mu_{B}$} & \multicolumn{3}{c}{Band gap (eV)}  \\
		\cline{4-9}
		& & & Cr(1)& Cr(2) & Tot  & $\uparrow$ & $\downarrow$ &  Tot \\
		\midrule
		U=0.0 & 6.91 & -48.442 & 3.13 & 3.13 & 5.79 & 0.58 & 0.38 & 0.36 \\
		U=0.5 & 6.91 & -47.804 & 3.21 & 3.21 & 5.87 & 0.54 & 0.46 & 0.43 \\
		U=1.0 & 6.92 & -47.190 & 3.29 & 3.29 & 5.95 & 0.51 & 0.54 & 0.50 \\
		U=1.5 & 6.93 & -46.598 & 3.37 & 3.37 & 6.03 & 0.47 & 0.62 & 0.47 \\
		U=2.0 & 6.94 & -46.029 & 3.45 & 3.45 & 6.11 & 0.43 & 0.70 & 0.43 \\
		U=2.5 & 6.95 & -45.482 & 3.53 & 3.53 & 6.18 & 0.39 & 0.77 & 0.39 \\
		U=3.0 & 6.95 & -44.957 & 3.61 & 3.61 & 6.26 & 0.35 & 0.84 & 0.35 \\
		U=3.5 & 6.96 & -44.455 & 3.68 & 3.68 & 6.33 & 0.31 & 0.90 & 0.31 \\
		U=4.0 & 6.97 & -43.973 & 3.75 & 3.75 & 6.40 & 0.27 & 0.96 & 0.27 \\
		\midrule
		\rowcolor{WhiteSmoke!70!Lavender}
		Magnetic phase & \multicolumn{8}{c}{Anti-ferromagnetic} \\
		\midrule
		\multirow{2}{*}{Properties} & \multirow{2}{*}{$a=b$(\AA)}& \multirow{2}{*}{Total E(eV)} & \multicolumn{3}{c}{$\mu_{B}$} & \multicolumn{3}{c}{Band gap (eV)}  \\
		\cline{4-9}
		& & & Cr(1)& Cr(2) & Tot  & $\uparrow$ & $\downarrow$ &  Tot \\
		\midrule
		U=0.0 & 6.86 & -48.368 & 3.04 & -3.04 & 0.00 & 0.50 & 0.50 & 0.50 \\
		U=0.5 & 6.87 & -47.724 & 3.13 & -3.13 & 0.00 & 0.54 & 0.54 & 0.54 \\
		U=1.0 & 6.88 & -47.104 & 3.22 & -3.22 & 0.00 & 0.57 & 0.57 & 0.57 \\
		U=1.5 & 6.89 & -46.509 & 3.31 & -3.31 & 0.00 & 0.59 & 0.59 & 0.59 \\
		U=2.0 & 6.90 & -45.937 & 3.40 & -3.40 & 0.00 & 0.59 & 0.59 & 0.59 \\
		U=2.5 & 6.91 & -45.389 & 3.48 & -3.48 & 0.00 & 0.56 & 0.56 & 0.56 \\
		U=3.0 & 6.92 & -44.863 & 3.57 & -3.57 & 0.00 & 0.52 & 0.52 & 0.52 \\
		U=3.5 & 6.93 & -44.360 & 3.65 & -3.65 & 0.00 & 0.49 & 0.49 & 0.49 \\
		U=4.0 & 6.94 & -43.878 & 3.72 & -3.72 & 0.00 & 0.46 & 0.46 & 0.46 \\
		\bottomrule
		\bottomrule
		\label{tab:4.3}
	\end{tabular}
\end{table}




\begin{table}[H]
	\centering
	\sisetup{table-format=3.0, table-number-alignment=center, group-separator={,}}
	\setlength{\extrarowheight}{0.5ex}
	\caption{Calculated electronic and magnetic properties for FM and AFM phases of CrGeTe$_3$ monolayer using the PBESol functional. Both magnetic phases exhibit semiconductor behavior and maintain symmetry conservation (space group 147), similar to the CGT monolayer.}
	\begin{tabular}{cccccc|ccc}
		\toprule
		\toprule
		\rowcolor{WhiteSmoke!70!Lavender}
		Magnetic phase & \multicolumn{8}{c}{Ferromagnetic} \\
		\midrule
		\multirow{2}{*}{Properties} & \multirow{2}{*}{$a=b$(\AA)}& \multirow{2}{*}{Total E(eV)} & \multicolumn{3}{c}{$\mu_{B}$} & \multicolumn{3}{c}{Band gap (eV)}  \\
		\cline{4-9}
		& & & Cr(1)& Cr(2) & Tot  & $\uparrow$ & $\downarrow$ &  Tot \\
		\midrule
		U=0.0 & 6.79 & -52.230 & 3.06 & 3.06 & 5.74 & 0.55 & 0.18 & 0.18 \\
		U=0.5 & 6.80 & -51.568 & 3.14 & 3.14 & 5.82 & 0.51 & 0.26 & 0.26 \\
		U=1.0 & 6.80 & -50.928 & 3.22 & 3.22 & 5.89 & 0.48 & 0.34 & 0.34 \\
		U=1.5 & 6.81 & -50.312 & 3.30 & 3.30 & 5.97 & 0.44 & 0.42 & 0.40 \\
		U=2.0 & 6.82 & -49.717 & 3.38 & 3.38 & 6.05 & 0.40 & 0.50 & 0.40 \\
		U=2.5 & 6.82 & -49.146 & 3.46 & 3.46 & 6.13 & 0.36 & 0.57 & 0.36 \\
		U=3.0 & 6.83 & -48.597 & 3.54 & 3.54 & 6.20 & 0.32 & 0.65 & 0.32 \\
		U=3.5 & 6.84 & -48.070 & 3.62 & 3.62 & 6.28 & 0.28 & 0.71 & 0.28 \\
		U=4.0 & 6.84 & -47.565 & 3.69 & 3.69 & 6.35 & 0.24 & 0.78 & 0.24 \\
		\midrule
		\rowcolor{WhiteSmoke!70!Lavender}
		Magnetic phase & \multicolumn{8}{c}{Anti-ferromagnetic} \\
		\midrule
		\multirow{2}{*}{Properties} & \multirow{2}{*}{$a=b$(\AA)}& \multirow{2}{*}{Total E(eV)} & \multicolumn{3}{c}{$\mu_{B}$} & \multicolumn{3}{c}{Band gap (eV)}  \\
		\cline{4-9}
		& & & Cr(1)& Cr(2) & Tot  & $\uparrow$ & $\downarrow$ &  Tot \\
		\midrule
		U=0.0 & 6.74 & -52.166 & 2.97 & -2.97 & 0.00 & 0.35 & 0.35 & 0.35 \\
		U=0.5 & 6.75 & -51.495 & 3.06 & -3.06 & 0.00 & 0.40 & 0.40 & 0.40 \\
		U=1.0 & 6.76 & -50.849 & 3.15 & -3.15 & 0.00 & 0.44 & 0.44 & 0.44 \\
		U=1.5 & 6.76 & -50.228 & 3.24 & -3.24 & 0.00 & 0.48 & 0.48 & 0.48 \\
		U=2.0 & 6.77 & -49.631 & 3.33 & -3.33 & 0.00 & 0.50 & 0.50 & 0.50 \\
		U=2.5 & 6.78 & -49.058 & 3.42 & -3.42 & 0.00 & 0.49 & 0.49 & 0.49 \\
		U=3.0 & 6.79 & -48.509 & 3.50 & -3.50 & 0.00 & 0.46 & 0.46 & 0.46 \\
		U=3.5 & 6.80 & -47.982 & 3.58 & -3.58 & 0.00 & 0.42 & 0.42 & 0.42 \\
		U=4.0 & 6.80 & -47.478 & 3.66 & -3.66 & 0.00 & 0.39 & 0.39 & 0.39 \\
		\bottomrule
		\bottomrule
		\label{tab:4.4}
	\end{tabular}
\end{table}

After performing extensive density functional theory (DFT) calculations to determine the electronic and magnetic properties of the ferromagnetic and antiferromagnetic phases, using two different functionals, we summarize the results as follows:

\begin{figure}[H]
	\centering
	\begin{subfigure}{.50\textwidth}
		\centering
		\includegraphics[width=.95\linewidth]{Figures/lattice_cgt_comparison.png}
	\end{subfigure}%
	\hfill % Fill space between subfigures
	\begin{subfigure}{.50\textwidth}
		\centering
		\includegraphics[width=.95\linewidth]{Figures/magnetic_cgt_comparison.png}
	\end{subfigure}
	\begin{subfigure}{.60\textwidth}
		\centering
		\includegraphics[width=\linewidth]{Figures/gap_cgt_comparison.png}
	\end{subfigure}
	\caption{Main lattice and magnetic parameters as a function of the Hubbard $U$ correction. In all plots, the $x$ axis represents the Hubbard $U$ values. ($\mathbf{a}$) Error analysis of lattice parameters compared to experimental values, using PBE and PBESol functionals. The PBESol functional with a Hubbard $U$ value of 3.0 eV provides the highest accuracy, showing a relative error of 0$\%$. ($\mathbf{b}$) Magnetic moment dependence on the Hubbard $U$ parameter using both functionals. ($\mathbf{c}$) Comparison of band gap corrections with PBE and PBESol functionals using Hubbard $U$ corrections. The addition of $U$ significantly enhances the band gap compared to the uncorrected cases. Although PBE yields a better improvement in the band gap, PBESol+$U=3.0$ is selected for further studies due to its superior accuracy in reproducing electronic properties.}
	\label{fig:4.7}
\end{figure}

The results confirm that the PBESol functional with a Hubbard $U$ correction of $U_{\text{eff}} = 3.0$ reproduces the lattice parameters more accurately compared to the PBE functional. This is because the PBE functional, which includes a stronger density-gradient dependence, tends to overestimate the lattice constants. While PBE is more appropriate for determining ground state energies, the choice of Hubbard $U_{\text{eff}} = 3.0$ is further validated by the improvement in the calculated band gap. Despite the absence of experimental band gap values for direct comparison, several theoretical studies using advanced functionals provide reference values. For example, using the HSE06 hybrid functional, we obtain a band gap of 0.68 eV. Incorporating spin-orbit coupling (SOC) with HSE06 yields an even lower value of 0.33 eV \supercite{Fang2018}. With $U = 3.0$, our calculated band gap shows a relative error of only 3.13$\%$, which is significantly smaller than the 15.63$\%$ and 43.75$\%$ errors observed for the PBE and PBESol functionals, respectively, without Hubbard $U$. See Fig. \ref{fig:cgtpbesolucorrection} for further details.

\begin{figure}[H]
	\begin{subfigure}{.5\textwidth}
		\centering
		\includegraphics[width=1\linewidth]{Figures/bandsplot_cgt_fm_pbesol_u3.0.png}
	\end{subfigure}%
	\begin{minipage}{.5\textwidth}
		\vspace{-12.6cm}
		\centering
		\includegraphics[width=1\linewidth]{Figures/dosplot_cgt_fm_pbesol_u3.0.png}
		\captionsetup{justification=centering}
	\end{minipage}
	\caption{Band structure (left) and Density of States (DOS) plot (right) for the ferromagnetic CGT monolayer using the PBESol+$U(3.0)$ functional. The DOS plot shows well-localized $t_{2g}$ and $e_{g}$ orbitals.}
	\label{fig:cgtpbesolucorrection}
\end{figure}


\subsection{Phonon Band Structure}
To accurately assess the stability of the CrGeTe$_3$ monolayer, phonon calculations are essential. Phonons, which represent quantized lattice vibrations, play a key role in understanding the vibrational properties of crystal structures. These calculations provide detailed information about the vibrational modes of the lattice, offering insights into the stability, potential instabilities, or phase transitions of the material. Using a 3x3x1 ferromagnetic supercell and the PBESol$+U(3.0)$ functional, the resulting phonon band structure is presented in Fig. \ref{fig:4.8}.

\begin{figure}[H]
	\centering
	\includegraphics[width=\linewidth]{Figures/pho_bands_dos_fm_cgt.png}
	\caption{Phonon band structure (left) for the ferromagnetic CGT monolayer (3x3x1 supercell) and Density of States (DOS) plot (right). The absence of negative frequencies, along with the presence of acoustic phonons originating from the high-symmetry point $\Gamma$, confirms the stability of the CGT monolayer in its ferromagnetic phase.}
	\label{fig:4.8}
\end{figure}

\section{MnGeTe\texorpdfstring{$_3$}{} Monolayer}
In this section, we present our findings on the MnGeTe$_3$ monolayer. This system shares the same symmetry (space group 147) as the CrGeTe$_3$ monolayer, but with manganese as the transition metal. Experimental studies on this material are yet to be reported, and to date, only theoretical studies have been carried out \supercite{Chittari2020, Song2023, Hao2021}.

\subsection{Electronic Properties Using the PBE Functional}
To investigate the MnGeTe$_3$ monolayer, we employed the PBE functional. After structural relaxation of the unit cell in both ferromagnetic (FM) and antiferromagnetic (AFM) configurations, we calculated key electronic properties such as lattice parameters, magnetic moments, and total energy. These results are summarized in Figure \ref{fig:4.9}.

\begin{figure}[H]
	\begin{minipage}[b]{.45\linewidth}
		\centering
		\includegraphics[width=0.9\linewidth]{Figures/eqs_mgtpbe.png}
		\vspace{-5cm}
		\captionof{figure}{The left panel displays the two-dimensional equation of state for the MnGeTe$_3$ monolayer in both FM and AFM phases. The FM phase is more stable with $\Delta E = E_{AFM} - E_{FM} = 0.0014$ eV/atom. The right panel shows the calculated electronic and magnetic properties for the FM and AFM phases of MnGeTe$_3$ ML using the PBE functional. Both phases exhibit metallic behavior, maintaining the symmetry of space group 147, similar to the CrGeTe$_3$ monolayer.}
		\label{fig:4.9}
	\end{minipage}\hfill
	\begin{minipage}[b]{.55\linewidth}
		\centering
		\resizebox{8cm}{!}{
			\begin{tabular}{ccccccc}
				\toprule
				\toprule
				Properties          & \multicolumn{3}{c}{FM phase} & \multicolumn{3}{c}{AFM phase} \\
				\midrule
				Space group         & \multicolumn{3}{c}{P-3 (147)}      & \multicolumn{3}{c}{P-3 (147)}       \\
				$a=b$ (\AA)         & \multicolumn{3}{c}{6.89}     & \multicolumn{3}{c}{6.92}      \\
				$c$  (\AA)          & \multicolumn{3}{c}{21.82}    & \multicolumn{3}{c}{21.82}     \\
				$\gamma$ (°)        & \multicolumn{3}{c}{120}      & \multicolumn{3}{c}{120}       \\
				Area (\AA$^2$)      & \multicolumn{3}{c}{41.17}    & \multicolumn{3}{c}{41.45}     \\
				Total Energy (eV)   & \multicolumn{3}{c}{-46.821}  & \multicolumn{3}{c}{-46.807}   \\
				\midrule
				\multirow{2}{*}{$\mu_{B}$} & Mn(1)  & Mn(2)  & Tot  &  Mn(1)  & Mn(2)  & Tot  \\
				&  3.44  & 3.44   & 6.49 &  3.47   & -3.47  &  0.00 \\
				\midrule
				\multirow{2}{*}{Band gap (eV)}  & $\uparrow$ & $\downarrow$ & Tot  & $\uparrow$ & $\downarrow$ & Tot \\
				&  0.00      & 0.00         & 0.00 & 0.00      & 0.00         & 0.00 \\
				\midrule
				sites & u      & v      & w       &   u    &    v    &   w     \\
				Mn(1) & 0.6666 & 0.3333 & 0.5000  & 0.6666 & 0.3333  & 0.5000 \\
				Mn(2) & 0.3333 & 0.6666 & 0.5000  & 0.3333 & 0.6666  & 0.5000 \\
				Ge(1) & 0.0000 & 0.0000 & 0.5568  & 0.0000 & 0.0000  & 0.5565 \\
				Ge(2) & 0.0000 & 0.0000 & 0.4432  & 0.0000 & 0.3333  & 0.4435 \\
				Te(1) & 0.0001 & 0.3780 & 0.4208  & 0.0001 & 0.3774  & 0.4211 \\
				Te(2) & 0.9999 & 0.6220 & 0.5792  & 0.9999 & 0.6226  & 0.5789 \\
				Te(3) & 0.6220 & 0.6221 & 0.4208  & 0.6226 & 0.6228  & 0.4211 \\
				Te(4) & 0.3780 & 0.3779 & 0.5792  & 0.3774 & 0.3772  & 0.5789 \\
				Te(5) & 0.3779 & 0.9999 & 0.4208  & 0.3772 & 0.9999  & 0.4211 \\
				Te(6) & 0.6221 & 0.0001 & 0.5792  & 0.6228 & 0.0001  & 0.5789 \\
				\bottomrule
				\bottomrule
			\end{tabular}
		}
	\end{minipage}
\end{figure}

As seen in Figure \ref{fig:4.9}, the energy difference between the FM and AFM phases is $\Delta E = 0.0014$ eV/atom. Although this value exceeds 1 meV/atom, it is still insufficient to conclusively establish the FM phase as definitively more stable. Given that the system is close to the stability threshold of 1 meV/atom, it raises the question of why the PBE functional, known for accurately predicting ground state energies, does not show a larger energy distinction between the magnetic phases. Therefore, further analysis of the MnGeTe$_3$ monolayer is needed. Nevertheless, we focused on the FM phase, which appears to be the more promising magnetic configuration.

We then computed the density of states (DOS) and band structure, as shown in Figure \ref{fig:4.10}. Despite the metallic nature revealed by the band structure in this FM phase, we hypothesize that MnGeTe$_3$ may exhibit half-metallic behavior. This inference is based on the few occupied states near the Fermi level in the spin-down channel, while the spin-up channel exhibits populated states near the Fermi level.

\begin{figure}[H]
	\begin{subfigure}{.5\textwidth}
		\centering
		\includegraphics[width=1\linewidth]{Figures/bandsplot_mgt_fm_pbe.png}
	\end{subfigure}%
	\begin{minipage}{.5\textwidth}
		\vspace{-12.75cm}
		\centering
		\includegraphics[width=1\linewidth]{Figures/dosplot_mgt_fm_pbe.png}
		\captionsetup{justification=centering}
	\end{minipage}
	\caption{Band structure (left) and Density of States (DOS) plot (right) for the MnGeTe$_3$ ML using the PBE functional. Significant localized $e_g$ states of Mn atoms are observed in the spin-down channel, while there is hybridization between $t_{2g}$ and $e_g$ states of Mn and $p$ states of Te in the spin-up channel.}
	\label{fig:4.10}
\end{figure}


\subsection{Electronic Properties Using the PBESol Functional}

We employed the PBESol functional, which provides a more accurate lattice parameter approximation compared to the standard PBE functional. Although no experimental data is available for this specific monolayer, the calculated results are shown in Figure \ref{fig:4.11}. The PBESol functional slightly underestimates the magnetic moment relative to PBE. Additionally, while the ferromagnetic (FM) phase is energetically more favorable, the energy difference between the FM and antiferromagnetic (AFM) phases is minimal ($\Delta E = 0.007$ eV/atom), similar to the results obtained with the PBE functional. Consequently, we focus our analysis on the FM phase.

\begin{figure}[H]
	\begin{minipage}[b]{.45\linewidth}
		\centering
		\includegraphics[width=0.9\linewidth]{Figures/eqs_mgtpbesol.png}
		\vspace{-5cm}
		\captionof{figure}{The left panel displays the two-dimensional equation of state for the MnGeTe$_3$ monolayer in both FM and AFM phases. The FM phase is slightly more stable with $\Delta E = E_{\text{AFM}} - E_{\text{FM}} = 0.007$ eV/atom. The right panel shows the electronic and magnetic properties for both phases using the PBESol functional. Both phases exhibit metallic behavior and retain the symmetry of space group 147, similar to the CrGeTe$_3$ monolayer.}
		\label{fig:4.11}
	\end{minipage}\hfill
	\begin{minipage}[b]{.55\linewidth}
		\centering
		\resizebox{8cm}{!}{
			\begin{tabular}{ccccccc}
				\toprule
				\toprule
				Properties          & \multicolumn{3}{c}{FM phase} & \multicolumn{3}{c}{AFM phase} \\
				\midrule
				Space group         & \multicolumn{3}{c}{P-3 (147)}      & \multicolumn{3}{c}{P-3 (147)}       \\
				$a=b$ (\AA)         & \multicolumn{3}{c}{6.76}     & \multicolumn{3}{c}{6.79}      \\
				$c$  (\AA)          & \multicolumn{3}{c}{21.82}    & \multicolumn{3}{c}{21.82}     \\
				$\gamma$(°)         & \multicolumn{3}{c}{120}      & \multicolumn{3}{c}{120}       \\
				Area (\AA$^2$)      & \multicolumn{3}{c}{39.62}    & \multicolumn{3}{c}{39.94}     \\
				Total Energy ($eV$) & \multicolumn{3}{c}{-50.663}  & \multicolumn{3}{c}{-50.649}   \\
				\midrule
				\multirow{2}{*}{$\mu_{B}$} & Mn(1) & Mn(2) & Total & Mn(1) & Mn(2) & Total \\
				&  3.20 & 3.20 & 6.07 &  3.05 & -3.05 &  0.00  \\ 
				\midrule
				\multirow{2}{*}{Band gap ($eV$)}  & $\uparrow$ & $\downarrow$ & Total &  $\uparrow$ & $\downarrow$ & Total \\
				&  0.00 & 0.00 &  0.00 &  0.00 &  0.00 &  0.00  \\ 
				\midrule                                                              
				Sites & u & v & w & u & v & w \\
				Mn(1) & 0.6666 & 0.3333 & 0.5000 & 0.6666 & 0.3333 & 0.5000 \\
				Mn(2) & 0.3333 & 0.6666 & 0.5000 & 0.3333 & 0.6666 & 0.5000 \\
				Ge(1) & 0.0000 & 0.0000 & 0.5563 & 0.0000 & 0.0000 & 0.5561 \\ 
				Ge(2) & 0.0000 & 0.0000 & 0.4437 & 0.0000 & 0.0000 & 0.4439 \\
				Te(1) & 0.0000 & 0.3824 & 0.4225 & 0.0000 & 0.3838 & 0.4242 \\
				Te(2) & 0.9999 & 0.6176 & 0.5775 & 0.9999 & 0.6162 & 0.5758 \\
				Te(3) & 0.6176 & 0.6176 & 0.4225 & 0.6162 & 0.6162 & 0.4242 \\
				Te(4) & 0.3824 & 0.3824 & 0.5775 & 0.3838 & 0.3838 & 0.5758 \\
				Te(5) & 0.3824 & 0.9999 & 0.4225 & 0.3838 & 0.9999 & 0.4242 \\
				Te(6) & 0.6176 & 0.0000 & 0.5775 & 0.6162 & 0.0000 & 0.5758 \\
				\bottomrule
				\bottomrule
			\end{tabular}
		}
	\end{minipage}
\end{figure}

The density of states (DOS) and band structure plots for the FM phase, as shown in Figure \ref{fig:4.12}, further confirm that the system does not exhibit half-metallic behavior. Instead, the results point towards a conventional metallic character.

\begin{figure}[H]
	\begin{subfigure}{.5\textwidth}
		\centering
		\includegraphics[width=1\linewidth]{Figures/bandsplot_mgt_fm_pbesol.png}
	\end{subfigure}%
	\begin{minipage}{.5\textwidth}
		\vspace{-12.6cm}
		\centering
		\includegraphics[width=1\linewidth]{Figures/dosplot_mgt_fm_pbesol.png}
		\captionsetup{justification=centering}
	\end{minipage}
	\caption{Band Structure (left) and Density of States (DOS) plot (right) for the ferromagnetic MnGeTe$_3$ monolayer using the PBESol functional. The DOS plot shows that the $t_{2g}$ orbitals are more localized in the spin-down channel (above the Fermi level), while in the spin-up channel (below the Fermi level), these states are clearly hybridized with the $p$ orbitals of Te.}
	\label{fig:4.12}
\end{figure}



\subsection{Magnetic and electronic properties using PBE and PBESol functionals with Hubbard U corrections}

The results obtained with the PBE and PBESol functionals provide a partial description of the half-metallic behavior. To better capture the strongly localized $d$ orbitals of Mn, we employ Hubbard $U$ corrections. This approach refines the total energies, clarifying the stability of the ferromagnetic (FM) phase. Notably, the energy difference between the antiferromagnetic (AFM) and FM phases, defined as $\Delta E = E_{AFM} - E_{FM}$, is marginal at 1.14 meV/atom for both functionals, indicating that the FM phase is nearly degenerate.

Utilizing Dudarev's approach for accounting for Hubbard $U$ corrections, we aim to achieve a clearer FM ground state and a more accurate description of the half-metallic behavior. The results are summarized in Tables \ref{tab:4.7} and \ref{tab:4.8}.

\begin{table}[H]
	\centering
	\sisetup{table-format=3.0, table-number-alignment=center, group-separator={,}}
	\setlength{\extrarowheight}{0.5ex}
	\caption{Calculated electronic and magnetic properties for FM and AFM phases of MnGeTe$_3$ monolayers using the PBE functional with Hubbard $U$ corrections.}
	\begin{tabular}{cccccc|ccc}
		\toprule
		\toprule
		\rowcolor{WhiteSmoke!70!Lavender}
		Magnetic phase & \multicolumn{8}{c}{Ferromagnetic} \\
		\midrule
		\multirow{2}{*}{Properties} & \multirow{2}{*}{$a=b$(\AA)}& \multirow{2}{*}{Total E(eV)} & \multicolumn{3}{c}{$\mu_{B}$} & \multicolumn{3}{c}{Band gap (eV)}  \\
		\cline{4-9}
		& & & Mn(1)& Mn(2) & Tot  & $\uparrow$ & $\downarrow$ &  Tot \\
		\midrule
		U=0.0 & 6.89 & -46.830 & 3.43 & 3.43 & 6.46 & 0.00 & 0.00 & 0.00 \\
		U=0.5 & 6.93 & -46.269 & 3.89 & 3.89 & 7.51 & 0.00 & 0.34 & 0.00 \\
		U=1.0 & 6.94 & -45.795 & 3.99 & 3.99 & 7.61 & 0.00 & 0.44 & 0.00 \\
		U=1.5 & 6.94 & -45.354 & 4.09 & 4.09 & 7.69 & 0.00 & 0.53 & 0.00 \\
		U=2.0 & 6.95 & -44.944 & 4.17 & 4.17 & 7.78 & 0.00 & 0.63 & 0.00 \\
		U=2.5 & 6.96 & -44.565 & 4.25 & 4.25 & 7.85 & 0.00 & 0.71 & 0.00 \\
		U=3.0 & 6.96 & -44.213 & 4.31 & 4.31 & 7.92 & 0.00 & 0.79 & 0.00 \\
		U=3.5 & 6.97 & -43.886 & 4.38 & 4.38 & 7.98 & 0.00 & 0.86 & 0.00 \\
		U=4.0 & 6.97 & -43.582 & 4.43 & 4.43 & 8.04 & 0.00 & 0.92 & 0.00 \\
		U=4.5 & 6.98 & -43.299 & 4.48 & 4.48 & 8.09 & 0.00 & 0.98 & 0.00 \\
		\midrule
		\rowcolor{WhiteSmoke!70!Lavender}
		Magnetic phase & \multicolumn{8}{c}{Anti-ferromagnetic} \\
		\midrule
		\multirow{2}{*}{Properties} & \multirow{2}{*}{$a=b$(\AA)}& \multirow{2}{*}{Total E(eV)} & \multicolumn{3}{c}{$\mu_{B}$} & \multicolumn{3}{c}{Band gap (eV)}  \\
		\cline{4-9}
		& & & Mn(1)& Mn(2) & Tot  & $\uparrow$ & $\downarrow$ &  Tot \\
		\midrule
		U=0.0 & 6.92 & -46.812 & 3.44 & -3.44 & 0.00 & 0.00 & 0.00 & 0.00 \\
		U=0.5 & 6.91 & -46.231 & 3.76 & -3.76 & 0.00 & 0.00 & 0.00 & 0.00 \\
		U=1.0 & 6.91 & -45.723 & 3.91 & -3.91 & 0.00 & 0.00 & 0.00 & 0.00 \\
		U=1.5 & 6.91 & -45.266 & 4.04 & -4.04 & 0.00 & 0.00 & 0.00 & 0.00 \\
		U=2.0 & 6.91 & -44.849 & 4.14 & -4.14 & 0.00 & 0.00 & 0.00 & 0.00 \\
		U=2.5 & 6.91 & -44.465 & 4.23 & -4.23 & 0.00 & 0.00 & 0.00 & 0.00 \\
		U=3.0 & 6.92 & -44.113 & 4.31 & -4.31 & 0.00 & 0.00 & 0.00 & 0.00 \\
		U=3.5 & 6.92 & -43.787 & 4.37 & -4.37 & 0.00 & 0.00 & 0.00 & 0.00 \\
		U=4.0 & 6.92& -43.486 & 4.43 & -4.43 & 0.00 & 0.00 & 0.00 & 0.00 \\
		U=4.5 & 6.92 & -43.209 & 4.48 & -4.48 & 0.00 & 0.00 & 0.00 & 0.00 \\
		\bottomrule
		\bottomrule
		\label{tab:4.7}
	\end{tabular}
\end{table}





\begin{table}[H]
	\centering
	\sisetup{table-format=3.0, table-number-alignment=center, group-separator={,}}
	\setlength{\extrarowheight}{0.5ex}
	\caption{Calculated electronic and magnetic properties for FM and AFM phases of MnGeTe$_3$ monolayers using the PBESol functional with Hubbard $U$ corrections.}
	\begin{tabular}{cccccc|ccc}
		\toprule
		\toprule
		\rowcolor{WhiteSmoke!70!Lavender}
		Magnetic phase & \multicolumn{8}{c}{Ferromagnetic} \\
		\midrule
		\multirow{2}{*}{Properties}& \multirow{2}{*}{$a=b$(\AA)}& \multirow{2}{*}{Total E(eV)} & \multicolumn{3}{c}{$\mu_{B}$} & \multicolumn{3}{c}{Band gap (eV)}  \\
		\cline{4-9}
		&  & & Mn(1)& Mn(2) & Tot  & $\uparrow$ & $\downarrow$ &  Tot \\
		\midrule
		U=0.0 & 6.76 & -50.684 & 3.11 & 3.11 & 5.86 & 0.00 & 0.00 & 0.00 \\
		U=0.5 & 6.77 & -50.005 & 3.46 & 3.46 & 6.51 & 0.00 & 0.00 & 0.00 \\
		U=1.0 & 6.80 & -49.444 & 3.91 & 3.91 & 7.52 & 0.00 & 0.19 & 0.00 \\
		U=1.5 & 6.81 & -48.973 & 4.00 & 4.00 & 7.61 & 0.00 & 0.28 & 0.00 \\
		U=2.0 & 6.81 & -48.535 & 4.09 & 4.09 & 7.69 & 0.00 & 0.37 & 0.00 \\
		U=2.5 & 6.81 & -48.127 & 4.17 & 4.17 & 7.77 & 0.00 & 0.00 & 0.00 \\
		U=3.0 & 6.82 & -47.748 & 4.25 & 4.25 & 7.86 & 0.00 & 0.00 & 0.00 \\
		U=3.5 & 6.81 & -47.397 & 4.32 & 4.32 & 7.96 & 0.00 & 0.00 & 0.00 \\
		U=4.0 & 6.81 & -47.072 & 4.38 & 4.38 & 8.05 & 0.00 & 0.00 & 0.00 \\
		U=4.5 & 6.81 & -46.769 & 4.43 & 4.43 & 8.12 & 0.00 & 0.00 & 0.00 \\
		\midrule
		\rowcolor{WhiteSmoke!70!Lavender}
		Magnetic phase & \multicolumn{8}{c}{Anti-ferromagnetic} \\
		\midrule
		\multirow{2}{*}{Properties} & \multirow{2}{*}{$a=b$(\AA)}& \multirow{2}{*}{Total E(eV)} & \multicolumn{3}{c}{$\mu_{B}$} & \multicolumn{3}{c}{Band gap (eV)}  \\
		\cline{4-9}
		&  & & Mn(1)& Mn(2) & Tot  & $\uparrow$ & $\downarrow$ &  Tot \\
		\midrule
		U=0.0 & 6.79 & -50.670 & 2.95 & -2.95 & 0.00 & 0.00 & 0.00 & 0.00 \\
		U=0.5 & 6.80 & -49.974 & 3.47 & -3.47 & 0.00 & 0.00 & 0.00 & 0.00 \\
		U=1.0 & 6.79 & -49.401 & 3.78 & -3.78 & 0.00 & 0.00 & 0.00 & 0.00 \\
		U=1.5 & 6.77 & -48.898 & 3.94 & -3.94 & 0.00 & 0.00 & 0.00 & 0.00 \\
		U=2.0 & 6.77 & -48.448 & 4.06 & -4.06 & 0.00 & 0.00 & 0.00 & 0.00 \\
		U=2.5 & 6.77 & -48.037 & 4.16 & -4.16 & 0.00 & 0.00 & 0.00 & 0.00 \\
		U=3.0 & 6.76 & -47.662 & 4.25 & -4.25 & 0.00 & 0.00 & 0.00 & 0.00 \\
		U=3.5 & 6.76 & -47.316 & 4.32 & -4.32 & 0.00 & 0.00 & 0.00 & 0.00 \\
		U=4.0 & 6.75 & -46.996 & 4.39 & -4.39 & 0.00 & 0.00 & 0.00 & 0.00 \\
		U=4.5 & 6.74 & -46.701 & 4.44 & -4.44 & 0.00 & 0.00 & 0.00 & 0.00 \\
		\bottomrule
		\bottomrule
		\label{tab:4.8}
	\end{tabular}
\end{table}

Those results are summarized in the following figure:

\begin{figure}[H]
	\centering
	\begin{subfigure}{.50\textwidth}
		\centering
		\includegraphics[width=.95\linewidth]{Figures/deltaUenergy_mgt.png}
	\end{subfigure}%
	\hfill % Fill space between subfigures
	\begin{subfigure}{.50\textwidth}
		\centering
		\includegraphics[width=.95\linewidth]{Figures/magnetic_mgt_comparison.png}
	\end{subfigure}
	
	\begin{subfigure}{.60\textwidth}
		\centering
		\includegraphics[width=\linewidth]{Figures/gap_mgt_comparison.png}
	\end{subfigure}
	\caption{\textbf{(a)} Energy difference between FM and AFM phases for PBE and PBESol functionals with various Hubbard $U_{eff} = U - J$ corrections ranging from 0 to 4.5 eV. The PBE functional exhibits a stronger FM phase with $U=2.5$ or $U=3.0$ eV compared to PBESol with similar Hubbard corrections, indicating improved accuracy in ground state energies using PBE. \textbf{(b)} Magnetic moments as a function of Hubbard $U$ parameters. \textbf{(c)} Band gap for both spin channels as a function of Hubbard $U$ corrections, demonstrating clear half-metallic (HM) behavior in the FM phase with PBE and Hubbard $U$ corrections. Notably, for PBEsol and Hubbard $U$ corrections, the band gap for the spin-down channel increases up to $U=2.0$ eV before decreasing, exhibiting metallic behavior at higher $U$ values.}
	\label{fig:4.13}
\end{figure}

Figure \ref{fig:4.13} illustrates that the PBE functional with Hubbard $U$ corrections effectively approximates the half-metallic behavior of the FM phase. However, it is essential to consider the magnetic moment, especially in the absence of experimental data. To address this, we employ the hybrid HSE06 functional, which predicts a magnetic moment of $4.23 \ \mu_{B}$ and a spin-down band gap of $1.47$ eV. Although Hubbard $U$ corrections can approximate this band gap, as demonstrated in Figure \ref{fig:4.11} (a), they do not necessarily stabilize the FM phase. Notably, a Hubbard $U$ value of $2.5$ eV provides the closest match to the HSE06 magnetic moment, yielding a minimal relative error of $0.47\%$.

\begin{figure}[H]
	\begin{subfigure}{.5\textwidth}
		\centering
		\includegraphics[width=1\linewidth]{Figures/bandsplot_mgt_fm_pbe_u2.5.png}
	\end{subfigure}%
	\begin{minipage}{.5\textwidth}
		\vspace{-12.4cm}
		\centering
		\includegraphics[width=1\linewidth]{Figures/dosplot_mgt_fm_pbe_u2.5.png}
		\captionsetup{justification=centering}
	\end{minipage}
	\caption{Band structure (left) and density of states (DOS) plot (right) for the ferromagnetic MGT monolayer using the PBE$+U(2.5)$ functional. The red and blue colors represent spin-up and spin-down channels in the band structure plot. The DOS plot clearly indicates well-behaved localization for both $t_{2g}$ and $e_{g}$ orbitals, demonstrating clear half-metallic behavior. The spin-down channel exhibits semiconductor behavior, while the spin-up channel shows metallic characteristics.}
	\label{fig:mgtpbeucorrection}
\end{figure}


\subsection{Phonon Band Structure}

Building on the previous analysis, we now investigate the phonon properties to assess the thermal stability of the ferromagnetic (FM) phase using the PBE+U(2.5) functional. Phonon calculations were performed using a 4x4x1 supercell to ensure accuracy and capture potential instabilities. The phonon density of states (DOS) and band structure, illustrated in Figure \ref{fig:4.14}, indicate that the system exhibits no negative frequencies, signifying dynamic stability. The phonon band structure (left) confirms the presence of acoustic phonons originating from the high-symmetry point $\Gamma$, further supporting the stability of the ferromagnetic phase of the monolayer.

This stability analysis complements our previous findings, reinforcing that the PBE+U(2.5) functional not only provides an accurate description of the electronic properties but also predicts a stable FM phase under thermal conditions.

\begin{figure}[H]
	\centering
	\includegraphics[width=\linewidth]{Figures/pho_bands_dos_fm_mgt.png}
	\caption{Phonon band structure (left) and phonon density of states (DOS) (right) for the ferromagnetic MGT monolayer with a 4x4x1 supercell.}
	\label{fig:4.14}
\end{figure}


\section{FeGeTe\texorpdfstring{$_3$}{} Monolayer}

\subsection{Electronic Properties Using PBE Functional}

Finally, we present the analysis of the FeGeTe\(_3\) monolayer, initially studied using the PBE functional. In our calculations, the lattice parameters were optimized, resulting in a change of symmetry from space group 147 to 162 for both magnetic phases (see Figure \ref{fig:4.15}). This adjustment indicates unusual magnetic behavior in the monolayer, warranting further investigation.

\begin{figure}[H]
	\begin{minipage}[b]{.45\linewidth}
		\centering
		\includegraphics[width=0.9\linewidth]{Figures/eqs_fgtpbe.png}
		\vspace{-5cm}
		\captionof{figure}{Electronic and magnetic properties of FeGeTe$_3$ monolayer for ferromagnetic (FM) and antiferromagnetic (AFM) phases using the PBE functional. The energy difference $\Delta E = -0.026 \, \text{eV/f.u.}$ suggests that the antiferromagnetic (AFM) phase may be more stable.}
		\label{fig:4.15}
	\end{minipage}\hfill
	\begin{minipage}[b]{.55\linewidth}
		\centering
		\resizebox{8cm}{!}{
			\begin{tabular}{ccccccc}
				\toprule
				\toprule
				Properties          & \multicolumn{3}{c}{FM phase} & \multicolumn{3}{c}{AFM phase} \\
				\midrule
				Space Group         & \multicolumn{3}{c}{P-3 (162)}      & \multicolumn{3}{c}{P-3 (162)}       \\
				$a=b$ (\AA)         & \multicolumn{3}{c}{6.82}     & \multicolumn{3}{c}{6.85}      \\
				$c$  (\AA)          & \multicolumn{3}{c}{21.82}    & \multicolumn{3}{c}{21.82}     \\
				$\gamma$(°)         & \multicolumn{3}{c}{120}      & \multicolumn{3}{c}{120}       \\
				Area (\AA)          & \multicolumn{3}{c}{40.33}    & \multicolumn{3}{c}{40.57}     \\
				Total Energy ($eV$) & \multicolumn{3}{c}{-44.400}   & \multicolumn{3}{c}{-44.451}    \\
				\midrule
				\multirow{2}{*}{$\mu_{B}$}        & Fe(1)      & Fe(2)        & Tot   &  Fe(1)      & Fe(2)        & Tot  \\
				& 1.21       & 1.21         & 2.12  &  1.38       & -1.38        &  0.00  \\ 
				\midrule
				\multirow{2}{*}{Band Gap ($eV$)}  & $\uparrow$ & $\downarrow$ &  Tot  &  $\uparrow$ & $\downarrow$ & Tot  \\
				&  0.39      & 0.00         &  0.00 &   0.23      &  0.23        &  0.00  \\ 
				\midrule                                                              
				Sites & u      & v      & w       &   u    &    v    &   w     \\
				Fe(1) & 0.6666 & 0.3333 & 0.5000  & 0.6666 &  0.3333 & 0.5000  \\
				Fe(2) & 0.3333 & 0.6666 & 0.5000  & 0.3333 &  0.6666 & 0.5000  \\
				Ge(1) & 0.0000 & 0.0000 & 0.5562  & 0.0000 &  0.0000 & 0.5563  \\ 
				Ge(2) & 0.0000 & 0.0000 & 0.4438  & 0.0000 &  0.0000 & 0.4437  \\
				Te(1) & 0.0000 & 0.3866 & 0.4287  & 0.0000 &  0.3866 & 0.4291  \\
				Te(2) & 0.9999 & 0.6134 & 0.5713  & 0.9999 &  0.6134 & 0.5709  \\
				Te(3) & 0.6134 & 0.6134 & 0.4287  & 0.6134 &  0.6134 & 0.4291  \\
				Te(4) & 0.3865 & 0.3865 & 0.5713  & 0.3866 &  0.3866 & 0.5709  \\
				Te(5) & 0.3865 & 0.9999 & 0.4287  & 0.3866 &  0.9999 & 0.4291  \\
				Te(6) & 0.6134 & 0.0000 & 0.5713  & 0.6134 &  0.0000 & 0.5709  \\
				\bottomrule
				\bottomrule
			\end{tabular}
		}
	\end{minipage}
\end{figure}

The energy difference $\Delta E = E_{FM} - E_{AFM} = -0.026 \, \text{eV/f.u.}$ indicates a preference for the antiferromagnetic (AFM) phase over the ferromagnetic (FM) phase. This observation directs our focus toward the AFM phase for subsequent analysis.

In Figure \ref{fig:4.16}, we present the band structure and density of states (DOS) for the FeGeTe\(_3\) monolayer in the AFM phase. The DOS reveals a prominent delocalization of 'd' states and significant hybridization between Fe 'd' and Te 'p' orbitals. This hybridization is evident both below and above the Fermi level, influencing the electronic properties of the monolayer.

\begin{figure}[H]
	\begin{subfigure}{.5\textwidth}
		\centering
		\includegraphics[width=1\linewidth]{Figures/bandsplot_fgt_afm_pbe.png}
	\end{subfigure}%
	\begin{minipage}{.5\textwidth}
		\vspace{-12.9cm}
		\centering
		\includegraphics[width=1\linewidth]{Figures/dosplot_fgt_afm_pbe.png}
		\captionsetup{justification=centering}
	\end{minipage}
	\caption{Band structure (left) and density of states (DOS) plot (right) for the AFM phase of FeGeTe$_3$ monolayer using the PBE functional. The AFM phase exhibits semiconductor behavior.}
	\label{fig:4.16}
\end{figure}

\subsection{Electronic properties using PBESol functional}

We next investigate the electronic properties of the FeGeTe\(_3\) monolayer using the PBESol functional. The calculated energy difference, \(\Delta E = 0.002 \, \text{eV/f.u.}\), indicates a slight preference for ferromagnetic behavior (see Fig. \ref{fig:4.17}). However, this energy difference is notably smaller than that observed with the PBE functional, where an antiferromagnetic phase was determined to be more stable.

\begin{figure}[H]
	\begin{minipage}[b]{.45\linewidth}
		\centering
		\includegraphics[width=0.9\linewidth]{Figures/eqs_fgtpbesol.png}
		\vspace{-5cm}
		\captionof{figure}{Calculated electronic and magnetic properties for FM and AFM phases of FeGeTe$_3$ ML using the PBESol functional. Both magnetic phases exhibit metallic behavior.}
	\end{minipage}\hfill
	\begin{minipage}[b]{.55\linewidth}
		\centering
		\resizebox{8cm}{!}{
			\begin{tabular}{ccccccc}
				\toprule
				\toprule
				Properties          & \multicolumn{3}{c}{FM phase} & \multicolumn{3}{c}{AFM phase} \\
				\midrule
				Space group         & \multicolumn{3}{c}{P-3 (162)}      & \multicolumn{3}{c}{P-3 (162)}       \\
				\(a = b\) (\AA)    & \multicolumn{3}{c}{6.65}     & \multicolumn{3}{c}{6.72}      \\
				\(c\) (\AA)        & \multicolumn{3}{c}{21.82}    & \multicolumn{3}{c}{21.82}     \\
				\(\gamma\) (°)     & \multicolumn{3}{c}{120}      & \multicolumn{3}{c}{120}       \\
				Area (\AA)         & \multicolumn{3}{c}{38.29}    & \multicolumn{3}{c}{39.13}     \\
				Total Energy (\(eV\)) & \multicolumn{3}{c}{-48.667}  & \multicolumn{3}{c}{-48.664}    \\
				\midrule
				\multirow{2}{*}{$\mu_{B}$} & Fe(1) & Fe(2) & Tot   &  Fe(1) & Fe(2) & Tot  \\
				&  1.02  & 1.02  & 1.97  &  1.23  & -1.23 & 0.00  \\ 
				\midrule
				\multirow{2}{*}{Band gap (\(eV\))} & $\uparrow$ & $\downarrow$ & Tot  &  $\uparrow$ & $\downarrow$ & Tot  \\
				&  0.00  & 0.00  &  0.00 &  0.00  &  0.00  & 0.00  \\ 
				\midrule                                                              
				Sites & \(u\) & \(v\) & \(w\) & \(u\) & \(v\) & \(w\) \\
				Fe(1) & 0.6666 & 0.3333 & 0.5000  & 0.6666 & 0.3333 & 0.5000  \\
				Fe(2) & 0.3333 & 0.6666 & 0.5000  & 0.3333 & 0.6666 & 0.5000  \\
				Ge(1) & 0.0000 & 0.0000 & 0.5559  & 0.0000 & 0.0000 & 0.5559  \\ 
				Ge(2) & 0.0000 & 0.0000 & 0.4441  & 0.0000 & 0.0000 & 0.4440  \\
				Te(1) & 0.0000 & 0.3941 & 0.4288  & 0.9999 & 0.3894 & 0.4301  \\
				Te(2) & 0.9999 & 0.6059 & 0.5712  & 0.0000 & 0.6106 & 0.5699  \\
				Te(3) & 0.6059 & 0.6059 & 0.4288  & 0.6106 & 0.6106 & 0.4301  \\
				Te(4) & 0.3941 & 0.3941 & 0.5712  & 0.3894 & 0.3894 & 0.5699  \\
				Te(5) & 0.3941 & 0.9999 & 0.4288  & 0.3894 & 0.0000 & 0.4301  \\
				Te(6) & 0.6059 & 0.0000 & 0.5712  & 0.6106 & 0.9999 & 0.5699  \\
				\bottomrule
				\bottomrule
			\end{tabular}
		}
	\end{minipage}
	\label{fig:4.17}
\end{figure}

\subsection{Magnetic and electronic properties using PBE and PBESol functional with Hubbard U corrections}

In this section, we analyze the magnetic and electronic properties of the FeGeTe$_3$ monolayer by applying Hubbard $U$ corrections to the PBE and PBESol functionals. The initial results from the PBE functional suggested the stability of the antiferromagnetic (AFM) phase, prompting a detailed examination of this phase with Hubbard $U$ corrections.

\begin{table}[H]
	\centering
	\sisetup{table-format=3.0, table-number-alignment=center, group-separator={,}}
	\setlength{\extrarowheight}{0.5ex}
	\caption{Calculated electronic and magnetic properties for ferromagnetic (FM) and antiferromagnetic (AFM) phases of FeGeTe$_3$ ML using the PBE functional.}
	\begin{tabular}{cccccc|ccc}
		\toprule
		\toprule
		\rowcolor{WhiteSmoke!70!Lavender}
		Magnetic Phase & \multicolumn{8}{c}{Ferromagnetic} \\
		\midrule
		\multirow{2}{*}{Properties} & \multirow{2}{*}{$a=b$ (\AA)} & \multirow{2}{*}{Total E (eV)} & \multicolumn{3}{c}{$\mu_{B}$} & \multicolumn{3}{c}{Band Gap (eV)} \\
		\cline{4-9}
		& & & Fe(1) & Fe(2) & Tot & $\uparrow$ & $\downarrow$ & Tot \\
		\midrule
		U = 0.0 & 6.82 & -44.483 & 1.19 & 1.19 & 2.10 & 0.39 & 0.13 & 0.13 \\
		U = 0.5 & 6.83 & -43.629 & 1.26 & 1.26 & 2.15 & 0.36 & 0.46 & 0.31 \\
		U = 1.0 & 6.83 & -42.806 & 1.33 & 1.33 & 2.22 & 0.33 & 0.76 & 0.33 \\
		U = 1.5 & 6.83 & -42.013 & 1.44 & 1.44 & 2.31 & 0.27 & 0.83 & 0.27 \\
		U = 2.0 & 6.84 & -41.256 & 1.57 & 1.57 & 2.42 & 0.20 & 0.84 & 0.20 \\
		U = 2.5 & 6.85 & -40.542 & 1.73 & 1.73 & 2.55 & 0.11 & 0.85 & 0.11 \\
		U = 3.0 & 6.86 & -39.878 & 1.90 & 1.90 & 2.70 & 0.00 & 0.86 & 0.00 \\
		U = 3.5 & 6.96 & -39.826 & 3.71 & 3.71 & 8.52 & 0.00 & 0.00 & 0.00 \\
		U = 4.0 & 6.97 & -39.441 & 3.73 & 3.73 & 8.60 & 0.00 & 0.00 & 0.00 \\
		\midrule
		\rowcolor{WhiteSmoke!70!Lavender}
		Magnetic Phase & \multicolumn{8}{c}{Antiferromagnetic} \\
		\midrule
		\multirow{2}{*}{Properties} & \multirow{2}{*}{$a=b$ (\AA)} & \multirow{2}{*}{Total E (eV)} & \multicolumn{3}{c}{$\mu_{B}$} & \multicolumn{3}{c}{Band Gap (eV)} \\
		\cline{4-9}
		& & & Fe(1) & Fe(2) & Tot & $\uparrow$ & $\downarrow$ & Tot \\
		\midrule
		U = 0.0 & 6.84 & -44.526 & 1.34 & -1.34 & 0.00 & 0.27 & 0.27 & 0.27 \\
		U = 0.5 & 6.84 & -43.674 & 1.42 & -1.42 & 0.00 & 0.42 & 0.42 & 0.42 \\
		U = 1.0 & 6.85 & -42.853 & 1.52 & -1.52 & 0.00 & 0.49 & 0.49 & 0.49 \\
		U = 1.5 & 6.91 & -41.878 & 3.08 & -3.08 & 0.00 & 0.00 & 0.00 & 0.00 \\
		U = 2.0 & 6.99 & -41.263 & 3.54 & -3.54 & 0.00 & 0.15 & 0.15 & 0.15 \\
		U = 2.5 & 6.99 & -40.784 & 3.62 & -3.62 & 0.00 & 0.31 & 0.31 & 0.31 \\
		U = 3.0 & 6.99 & -40.347 & 3.67 & -3.67 & 0.00 & 0.42 & 0.42 & 0.42 \\
		U = 3.5 & 6.99 & -39.937 & 3.69 & -3.69 & 0.00 & 0.49 & 0.49 & 0.49 \\
		U = 4.0 & 6.99 & -39.545 & 3.71 & -3.71 & 0.00 & 0.51 & 0.51 & 0.51 \\
		\bottomrule
		\bottomrule
		\label{tab:4.11}
	\end{tabular}
\end{table}

\begin{table}[H]
	\centering
	\sisetup{table-format=3.0, table-number-alignment=center, group-separator={,}}
	\setlength{\extrarowheight}{0.5ex}
	\caption{Calculated electronic and magnetic properties for ferromagnetic (FM) and antiferromagnetic (AFM) phases of FeGeTe$_3$ ML using the PBESol functional.}
	\begin{tabular}{cccccc|ccc}
		\toprule
		\toprule
		\rowcolor{WhiteSmoke!70!Lavender}
		Magnetic Phase & \multicolumn{8}{c}{Ferromagnetic} \\
		\midrule
		\multirow{2}{*}{Properties} & \multirow{2}{*}{$a=b$ (\AA)} & \multirow{2}{*}{Total E (eV)} & \multicolumn{3}{c}{$\mu_{B}$} & \multicolumn{3}{c}{Band Gap (eV)} \\
		\cline{4-9}
		& & & Fe(1) & Fe(2) & Tot & $\uparrow$ & $\downarrow$ & Tot \\
		\midrule
		U = 0.0 & 6.65 & -48.702 & 1.00 & 1.00 & 1.95 & 0.00 & 0.00 & 0.00 \\
		U = 0.5 & 6.71 & -47.783 & 1.18 & 1.18 & 2.09 & 0.33 & 0.14 & 0.14 \\
		U = 1.0 & 6.72 & -46.929 & 1.24 & 1.24 & 2.14 & 0.31 & 0.42 & 0.31 \\
		U = 1.5 & 6.72 & -46.102 & 1.32 & 1.32 & 2.21 & 0.27 & 0.69 & 0.27 \\
		U = 2.0 & 6.72 & -45.302 & 1.40 & 1.40 & 2.28 & 0.23 & 0.78 & 0.23 \\
		U = 2.5 & 6.73 & -44.533 & 1.52 & 1.52 & 2.36 & 0.17 & 0.81 & 0.17 \\
		U = 3.0 & 6.74 & -43.779 & 1.63 & 1.63 & 2.45 & 0.12 & 0.84 & 0.12 \\
		U = 3.5 & 6.74 & -43.031 & 1.78 & 1.78 & 2.58 & 0.07 & 0.88 & 0.07 \\
		U = 4.0 & 6.74 & -42.276 & 1.84 & 1.84 & 2.64 & 0.03 & 0.93 & 0.03 \\
		\midrule
		\rowcolor{WhiteSmoke!70!Lavender}
		Magnetic Phase & \multicolumn{8}{c}{Antiferromagnetic} \\
		\midrule
		\multirow{2}{*}{Properties} & \multirow{2}{*}{$a=b$ (\AA)} & \multirow{2}{*}{Total E (eV)} & \multicolumn{3}{c}{$\mu_{B}$} & \multicolumn{3}{c}{Band Gap (eV)} \\
		\cline{4-9}
		& & & Fe(1) & Fe(2) & Tot & $\uparrow$ & $\downarrow$ & Tot \\
		\midrule
		U = 0.0 & 6.67 & -48.303 & 1.00 & -1.00 & 0.00 & 0.00 & 0.00 & 0.00 \\
		U = 0.5 & 6.67 & -47.386 & 1.00 & -1.00 & 0.00 & 0.13 & 0.13 & 0.13 \\
		U = 1.0 & 6.67 & -46.542 & 1.00 & -1.00 & 0.00 & 0.28 & 0.28 & 0.28 \\
		U = 1.5 & 6.67 & -45.733 & 1.00 & -1.00 & 0.00 & 0.37 & 0.37 & 0.37 \\
		U = 2.0 & 6.68 & -44.938 & 3.21 & -3.21 & 0.00 & 0.47 & 0.47 & 0.47 \\
		U = 2.5 & 6.68 & -44.294 & 3.20 & -3.20 & 0.00 & 0.59 & 0.59 & 0.59 \\
		U = 3.0 & 6.68 & -43.543 & 3.19 & -3.19 & 0.00 & 0.61 & 0.61 & 0.61 \\
		U = 3.5 & 6.68 & -42.943 & 3.23 & -3.23 & 0.00 & 0.65 & 0.65 & 0.65 \\
		U = 4.0 & 6.68 & -42.557 & 3.25 & -3.25 & 0.00 & 0.67 & 0.67 & 0.67 \\
		\bottomrule
		\bottomrule
		\label{tab:4.12}
	\end{tabular}
\end{table}

The results are summarized in the following figures:

\begin{figure}[H]
	\centering
	\begin{subfigure}{.50\textwidth}
		\centering
		\includegraphics[width=.95\linewidth]{Figures/deltaUenergy_fgt.png}
	\end{subfigure}%
	\hfill % Fill space between subfigures
	\begin{subfigure}{.50\textwidth}
		\centering
		\includegraphics[width=.95\linewidth]{Figures/magnetic_fgt_comparison.png}
	\end{subfigure}
	\begin{subfigure}{.60\textwidth}
		\centering
		\includegraphics[width=\linewidth]{Figures/gap_fgt_comparison.png}
	\end{subfigure}
	\caption{\textbf{(a)} Difference in total energy as a function of Hubbard $U$ for both ferromagnetic and antiferromagnetic phases. The antiferromagnetic phase exhibits enhanced stability at specific $U$ values, with $U=3.0$ identified as the most favorable. \textbf{(b)} Dependence of magnetic moment on Hubbard $U$. \textbf{(c)} Variation of the band gap with Hubbard $U$. The PBE+U(1.0) functional yields a suitable band gap for the antiferromagnetic phase, aligning with the desired semiconductor behavior, although it does not fully replicate the HSE06 results.}
	\label{fig:4.18}
\end{figure}

To achieve semiconductor behavior in the antiferromagnetic phase, we recommend the PBE+U(1.0) functional (see Fig. \ref{fig:fgtpbeucorrection}). This functional maintains the stability of the antiferromagnetic phase and provides a magnetic moment of approximately $1.46$ $\mu_{B}$, which closely resembles the results obtained with the HSE06 functional. However, attaining a band gap comparable to that from HSE06 remains challenging with Hubbard $U$ corrections, primarily due to the relationship between the magnetic moment and Hubbard $U$, as illustrated in the final figure.

\begin{figure}[H]
	\begin{subfigure}{.5\textwidth}
		\centering
		\includegraphics[width=1\linewidth]{Figures/bandsplot_fgt_afm_pbe_u1.0.png}
	\end{subfigure}%
	\begin{minipage}{.5\textwidth}
		\vspace{-12.6cm}
		\centering
		\includegraphics[width=1\linewidth]{Figures/dosplot_fgt_afm_pbe_u1.0.png}
		\captionsetup{justification=centering}
	\end{minipage}
	\caption{Band structure (left) and density of states (DOS) plot (right) for the antiferromagnetic FeGeTe$_3$ monolayer using the PBE functional. Notably, hybridization occurs between the 'd' orbitals of Fe and the 'p' orbitals of Te, with stronger interactions observed for the $e_{g}$ states compared to the $t_{2g}$ states.}
	\label{fig:fgtpbeucorrection}
\end{figure}


\subsection{Phonon Band Structure}

To evaluate the thermodynamic stability of the antiferromagnetic (AFM) phase using the PBE+U(1.0) functional, we performed phonon calculations on a 4x4x1 supercell. Despite the application of Hubbard $U$ corrections, the phonon band structure analysis reveals the presence of negative frequencies, indicating potential instability in the AFM phase.

\begin{figure}[H]
	\centering
	\includegraphics[width=\linewidth]{Figures/pho_bands_dos_afm_fgt.png}
	\caption{Phonon band structure (left) and phonon density of states (DOS) plot (right) for the AFM phase of the FGT 4x4x1 supercell.}
	\label{fig:4.19}
\end{figure}



\section{Random Alloys}

In this section, we present our results for random alloys obtained using the Special Quasirandom Structure (SQS) method for the sublattice systems $X = \text{Cr-Mn}$, $\text{Cr-Fe}$, and $\text{Fe-Mn}$ within a 4x3x1 supercell of $X\text{GeTe}_3$, which consists of 24 formula units (f.u.) and a total of 120 atoms. This composition includes 24 atoms of the sublattice $X$, which varies based on the partial occupations ($x = 0.25$, $0.50$, and $0.75$), 24 atoms of Ge, and 72 atoms of Te. It is important to note that the SQS obtained represents random configurations fitting the lattice sites of $\text{Cr}_{x}\text{Mn}_{1-x}\text{GeTe}_3$. 

We employed several relaxation loops—rather than performing a full relaxation—to achieve optimized random alloys. The relaxation process was structured as follows: (1) relax the cell size (area), (2) relax the cell size again, (3) relax ionic positions, (4) relax the cell size once more, and (5) relax ionic positions again.

\subsection{\texorpdfstring{Cr$_{1-x}$GeMn$_{x}$Te$_{3}$ Random Alloys}{Cr1-xGeMnxTe3 Random Alloys}}

We begin with the random alloys $\text{Cr}_{1-x}\text{GeMn}_{x}\text{Te}_3$ at concentrations $x = 0.25$, $0.50$, and $0.75$. Some important properties, such as the magnetic moment and total energy, are presented in Table \ref{tab:4.13}. Notably, at the concentration $x = 0.50$, throughout the entire relaxation loop (except in the third step), we observed an unusual error (standard deviation) in the mean magnetic moment of the Mn atoms. Specifically, one of the 12 Mn atoms (Mn atom number 10) exhibited a negative magnetic moment throughout the relaxation process (except in the third step), with values ranging from -3.707 to -3.610 $\mu_B$. Conversely, if we disregard this Mn atom with a negative magnetic moment, the average magnetic moment across the entire relaxation process ranges from 3.552 to 3.568 $\mu_B$, with a small standard deviation ranging from $\pm$ 0.008 to $\pm$ 0.017 $\mu_B$. 

This behavior can be attributed to the local atomic environment of the Te atoms surrounding each Mn atom. For the 11 Mn atoms, the average distance to the surrounding Te atoms is approximately 2.78 to 2.79 \AA, while the distance for Mn atom number 10 is 2.81 \AA. This discrepancy may lead to magnetic frustration—competing magnetic interactions between the Cr and Mn atoms—in the random alloy at the concentration of $x = 0.50$. Consequently, it is energetically favorable for the Mn atom to align antiferromagnetically with its neighbors.  

\begin{table}[H]
	\centering
	\sisetup{table-format=3.0, table-number-alignment=center, group-separator={,}}
	\setlength{\extrarowheight}{0.5ex}
	\caption{Magnetic moments and total energies at different stages of the relaxation cycle for each concentration $x = 0.25$, $0.50$, and $0.75$. An issue with the average magnetic moment of Mn atoms at the concentration of $x = 0.50$ is noted, as its standard deviation is considerable.}
	\resizebox{17cm}{!}{
		\begin{tabular}{cccccccc}
			\toprule
			\toprule
			\cellcolor{WhiteSmoke!70!Lavender} & \cellcolor{WhiteSmoke!70!Lavender} &  \cellcolor{WhiteSmoke!70!Lavender} & \multicolumn{5}{c}{\cellcolor{WhiteSmoke!70!Lavender}Relaxation}\\ 
			\cline{4-8}
			\multirow{-2}{*}{\cellcolor{WhiteSmoke!70!Lavender}Concentration $x \%$} & \multirow{-2}{*}{\cellcolor{WhiteSmoke!70!Lavender}Property} & \multirow{-2}{*}{\cellcolor{WhiteSmoke!70!Lavender}Atom} & \cellcolor{WhiteSmoke!70!Lavender}Step 1 & \cellcolor{WhiteSmoke!70!Lavender}Step 2 & \cellcolor{WhiteSmoke!70!Lavender}Step 3 & \cellcolor{WhiteSmoke!70!Lavender}Step 4 & \cellcolor{WhiteSmoke!70!Lavender}Step 5 \\
			\midrule
			\midrule
			\multirow{5}{*}{25} &                & Cr     & 3.122    $\pm$ 0.009  & 3.122   $\pm$ 0.009 & 3.101 $\pm$ 0.012     & 3.091 $\pm$ 0.012    & 3.095 $\pm$ 0.012 \\
			& Magnetic                                  & Ge    & 0.022    $\pm$ 0.007  & 0.022  $\pm$ 0.007  & 0.023 $\pm$ 0.007   & 0.022 $\pm$ 0.007    & 0.023 $\pm$ 0.007 \\ 
			& Moment ($\mu_B / \text{atom}$)   & Mn    & 3.539   $\pm$ 0.004  & 3.539  $\pm$ 0.004 & 3.586 $\pm$ 0.010  & 3.570 $\pm$ 0.017       & 3.571 $\pm$ 0.016 \\ 
			&                                                   & Te      & -0.073 $\pm$ 0.011   & -0.073 $\pm$ 0.011  & -0.072 $\pm$ 0.011   & -0.071 $\pm$ 0.011    & -0.072 $\pm$ 0.011 \\ 
			&                                                   & Total & 77.677                           & 77.677                         & 77.637                          & 77.618                          & 77.688 \\ 
			\midrule
			& Total Energy (eV)                   &            & -566.222                      & -566.222                     & -566.187                      & -566.226                     & -566.234 \\	 		                                           
			\midrule
			\midrule
			\multirow{5}{*}{50} & & Cr & 3.131 $\pm$ 0.012 & 3.131 $\pm$ 0.012 & 3.113 $\pm$ 0.017 & 3.105 $\pm$ 0.016 & 3.107 $\pm$ 0.016 \\
			& Magnetic & Ge & 0.028 $\pm$ 0.009 & 0.028 $\pm$ 0.009 & 0.030 $\pm$ 0.008 & 0.027 $\pm$ 0.008 & 0.026 $\pm$ 0.008 \\ 
			& Moment ($\mu_B / \text{atom}$) & Mn & 2.940 $\pm$ 2.060 & 2.940 $\pm$ 2.060 & 3.568 $\pm$ 0.017 & 2.954 $\pm$ 2.079 & 2.947 $\pm$ 2.096 \\ 
			& & Te & -0.078 $\pm$ 0.024 & -0.078 $\pm$ 0.024 & -0.082 $\pm$ 0.011 & -0.077 $\pm$ 0.023 & -0.077 $\pm$ 0.023 \\ 
			& & Total & 67.947 & 67.947 & 74.961 & 67.824 & 67.672 \\ 
			\midrule
			& Total Energy (eV) & & -570.772 & -570.773 & -570.691 & -570.741 & -570.756 \\ 		     
			\midrule
			\midrule
			\multirow{5}{*}{75} & & Cr & 3.141 $\pm$ 0.012 & 3.141 $\pm$ 0.012 & 3.130 $\pm$ 0.014 & 3.122 $\pm$ 0.014 & 3.126 $\pm$ 0.014 \\
			& Magnetic & Ge & 0.037 $\pm$ 0.008 & 0.037 $\pm$ 0.008 & 0.034 $\pm$ 0.008 & 0.034 $\pm$ 0.008 & 0.034 $\pm$ 0.008 \\ 
			& Moment ($\mu_B / \text{atom}$) & Mn & 3.529 $\pm$ 0.007 & 3.529 $\pm$ 0.007 & 3.525 $\pm$ 0.007 & 3.511 $\pm$ 0.007 & 3.510 $\pm$ 0.007 \\ 
			& & Te & -0.081 $\pm$ 0.008 & -0.081 $\pm$ 0.008 & -0.078 $\pm$ 0.009 & -0.078 $\pm$ 0.009 & -0.078 $\pm$ 0.009 \\ 
			& & Total & 64.209 & 64.209 & 64.208 & 64.196 & 64.200 \\ 
			\midrule
			& Total Energy (eV) & & -570.771 & -570.771 & -570.771 & -570.770 & -570.770 \\ 		     
			\bottomrule
			\bottomrule
	\end{tabular}}
	\label{tab:4.13}
\end{table}

Using the values obtained from the final relaxation loop, corresponding to the most stable configuration, we computed key thermodynamic properties such as the formation and mixing energies. For the formation energy, we considered the 24 formula units (f.u.) that make up the random alloy Cr$_{1-x}$GeMn$_{x}$Te$_{3}$:

\begin{align*}
	\Delta_f(\text{Cr}_{1-x}\text{GeMn}_{x}\text{Te}_{3}) &= E(\text{Cr}_{24(1-x)}\text{Ge}_{24}\text{Mn}_{24x}\text{Te}_{72}) \\
	&\quad - \left[ 24(1-x)E(\text{Cr}) + 24E(\text{Ge}) + 24xE(\text{Mn}) + 72E(\text{Te}) \right]
\end{align*}

Here, \(E(\text{Cr}_{24(1-x)}\text{Ge}_{24}\text{Mn}_{24x}\text{Te}_{72})\) represents the total energy of the alloy obtained from the final relaxation loop, as summarized in Table \ref{tab:4.13}. The terms \(E(\text{Cr})\), \(E(\text{Ge})\), \(E(\text{Mn})\), and \(E(\text{Te})\) correspond to the chemical potentials of the individual elements. These chemical potentials are calculated by performing full structural relaxations on the pure elemental unit cells and dividing the total energy of each unit cell by the number of atoms it contains. These values serve as reference energies for the elements in their most stable crystalline forms, which are critical for calculating the formation energy of the alloy.

For the mixing energy, which also considers the 24 formula units, accounting for the energetic cost or benefit of forming the alloy from its individual components, we use the following expression:

\begin{align*}
	\Delta E_{\text{mix}}(\text{Cr}_{1-x}\text{GeMn}_{x}\text{Te}_{3}) &= E(\text{Cr}_{24(1-x)}\text{Ge}_{24}\text{Mn}_{24x}\text{Te}_{72}) \\
	&\quad - \left[ 24(1-x)E(\text{CrGeTe}_{3}) + 24xE(\text{MnGeTe}_{3}) \right]
\end{align*}

Here, \(E(\text{CrGeTe}_{3})\) and \(E(\text{MnGeTe}_{3})\) denote the total energies of the unmixed monolayers containing chromium and manganese, respectively. These total energies are precomputed using the Perdew–Burke–Ernzerhof (PBE) functional, as illustrated in Figures \ref{fig:4.3} and \ref{fig:4.9}. Both the formation and mixing energies are derived from the total energies obtained after the final relaxation step, ensuring that the most stable configuration of the alloy is used for all subsequent calculations.

\begin{figure}[H]
	\centering
	\begin{subfigure}{.50\textwidth}
		\centering
		\includegraphics[width=.95\linewidth]{Figures/dos_alloyCrMn25.png}
	\end{subfigure}%
	\hfill % Fill space between subfigures
	\begin{subfigure}{.50\textwidth}
		\centering
		\includegraphics[width=.95\linewidth]{Figures/dos_alloyCrMn50.png}
	\end{subfigure}
	\begin{subfigure}{.60\textwidth}
		\centering
		\includegraphics[width=\linewidth]{Figures/dos_alloyCrMn75.png}
	\end{subfigure}
	\caption{Projected density of states of the random alloys, where the x-axis represents the energy (eV) range concerning the Fermi level (displayed as a red dashed line) for Cr$_{x}$GeMn$_{1-x}$Te$_{3}$ at ($\mathbf{a}$) $x=0.25$, ($\mathbf{b}$) $x=0.50$, and ($\mathbf{c}$) $x=0.75$. In ($\mathbf{a}$) and ($\mathbf{b}$), a strong hybridization of the $d$ orbitals of the magnetic atoms with the $p$ orbitals of Te is observed. In ($\mathbf{c}$), a weak localization of the $e_{g}$ states of Cr atoms is noted for the spin-down channel above the Fermi level.}
	\label{fig:dosplotCrMnalloys}
\end{figure}

The formation energies obtained for concentrations of $x=0.25$, $x=0.50$, and $x=0.75$ are -0.503, -0.540, and -0.597 eV/f.u., respectively. A similar approach was employed to calculate the mixing energies, resulting in values of 0.037, 0.014, and 0.015 eV/f.u., respectively. These values indicate a global stability for the random alloys $Cr_{x}$GeMn$_{1-x}$Te$_{3}$ at the studied concentrations. However, the positive values of the mixing energies suggest a complex internal structure, implying that there may exist a different ground magnetic phase than the ferromagnetic one that we considered.

Finally, we present in Fig. \ref{fig:dosplotCrMnalloys} the projected density of states for these concentrations. Notably, for $x=0.25$, the random alloy exhibits well-defined $d$ states for Mn atoms; however, the $d$ states for Cr atoms are significantly hybridized with the $p$ orbitals of Te atoms. For the concentration $x=0.50$, the $d$ states are not well-localized for both Cr and Mn atoms, as they are hybridized with the $p$ orbitals of Te. In contrast, for the concentration $x=0.75$, the $d$ orbitals of Cr atoms are well-localized, while the $d$ states of Mn atoms remain localized but hybridized with the $p$ states of Te atoms.


\subsection{\texorpdfstring{Cr$_{1-x}$GeFe$_{x}$Te$_{3}$ random alloys}{Cr1-xGeFexTe3 random alloys}}

In this section, we examine the random alloy of the form Cr$_{1-x}$GeFe$_{x}$Te$_{3}$. The magnetic moments and total energies for the different concentrations \( x = 0.25, 0.50, \) and \( 0.75 \) are presented in Tab. \ref{tab:4.14}. The calculations show negligible errors, indicated by small standard deviations, across the five relaxation loops. Using the total energies from the final loop (the fifth loop), we can calculate the formation energies and mixing energies similarly to our previous analysis for the Cr$_{1-x}$GeMn$_{x}$Te$_{3}$ alloy.

\begin{table}[H]
	\centering
	\sisetup{table-format=3.0, table-number-alignment=center, group-separator={,}}
	\setlength{\extrarowheight}{0.5ex}
	\caption{Average magnetic moments per atom and total energies of the random alloys Cr$_{1-x}$GeFe$_{x}$Te$_{3}$ at the concentrations \( x = 0.25, 0.50, \) and \( 0.75 \) for each stage of the implemented relaxation cycle. Note that the standard deviation values are minimal, ensuring realistic calculations.}
	\resizebox{17cm}{!}{
		\begin{tabular}{cccccccc}
			\toprule
			\toprule
			\cellcolor{WhiteSmoke!70!Lavender} & \cellcolor{WhiteSmoke!70!Lavender} &  \cellcolor{WhiteSmoke!70!Lavender} & \multicolumn{5}{c}{\cellcolor{WhiteSmoke!70!Lavender}Relaxation}\\ 
			\cline{4-8}
			\multirow{-2}{*}{\cellcolor{WhiteSmoke!70!Lavender}Concentration $x \%$} & \multirow{-2}{*}{\cellcolor{WhiteSmoke!70!Lavender}Property} & \multirow{-2}{*}{\cellcolor{WhiteSmoke!70!Lavender}Atom} & \cellcolor{WhiteSmoke!70!Lavender}Step 1 & \cellcolor{WhiteSmoke!70!Lavender}Step 2 & \cellcolor{WhiteSmoke!70!Lavender}Step 3 & \cellcolor{WhiteSmoke!70!Lavender}Step 4 & \cellcolor{WhiteSmoke!70!Lavender}Step 5 \\
			\midrule
			\midrule
			\multirow{5}{*}{25} & & Cr & 3.054 $\pm$ 0.01 & 3.054 $\pm$ 0.01 & 3.063 $\pm$ 0.011 & 3.061 $\pm$ 0.012 & 3.058 $\pm$ 0.01 \\
			& Magnetic & Ge & 0.002 $\pm$ 0.007 & 0.002 $\pm$ 0.007 & 0.005 $\pm$ 0.006 & 0.005 $\pm$ 0.006 & 0.005 $\pm$ 0.006 \\ 
			& moment ($\mu_B /atom$) & Fe & 2.237 $\pm$ 0.029 & 2.239 $\pm$ 0.029 & 2.218 $\pm$ 0.051 & 1.238 $\pm$ 0.057 & 1.237 $\pm$ 0.06 \\ 
			& & Te & -0.068 $\pm$ 0.011 & -0.068 $\pm$ 0.011 & -0.059 $\pm$ 0.013 & -0.057 $\pm$ 0.013 & -0.057 $\pm$ 0.013 \\ 
			& & Total & 53.741 & 53.743 & 36.971 & 36.966 & 36.675 \\ 
			\midrule
			& Total Energy (eV) & & -538.728 & -538.727 & -543.151 & -543.378 & -543.396 \\ 		                                           
			\midrule
			\midrule
			\multirow{5}{*}{50} & & Cr & 3.084 $\pm$ 0.018 & 3.084 $\pm$ 0.023 & 3.087 $\pm$ 0.02 & 3.084 $\pm$ 0.02 & 3.082 $\pm$ 0.02 \\
			& Magnetic & Ge & 0.011 $\pm$ 0.009 & 0.011 $\pm$ 0.009 & 0.013 $\pm$ 0.008 & 0.012 $\pm$ 0.009 & 0.012 $\pm$ 0.009 \\ 
			& moment ($\mu_B /atom$) & Fe & 2.23 $\pm$ 0.061 & 2.231 $\pm$ 0.061 & 1.307 $\pm$ 0.071 & 1.328 $\pm$ 0.079 & 1.33 $\pm$ 0.08 \\ 
			& & Te & -0.076 $\pm$ 0.016 & -0.076 $\pm$ 0.016 & -0.074 $\pm$ 0.015 & -0.074 $\pm$ 0.015 & -0.074 $\pm$ 0.015 \\ 
			& & Total & 58.194 & 58.209 & 47.577 & 47.902 & 47.906 \\ 
			\midrule
			& Total Energy (eV) & & -552.521 & -552.524 & -555.368 & -555.549 & -555.554 \\ 		     
			\midrule
			\midrule
			\multirow{5}{*}{75} & & Cr & 2.06 $\pm$ 1.491 & 2.06 $\pm$ 1.491 & 3.114 $\pm$ 0.019 & 3.111 $\pm$ 0.019 & 3.11 $\pm$ 0.019 \\
			& Magnetic & Ge & 0.564 $\pm$ 0.98 & 0.564 $\pm$ 0.98 & 0.025 $\pm$ 0.011 & 0.025 $\pm$ 0.012 & 0.025 $\pm$ 0.012 \\ 
			& moment ($\mu_B /atom$) & Fe & 2.238 $\pm$ 0.077 & 2.237 $\pm$ 0.078 & 1.426 $\pm$ 0.063 & 1.455 $\pm$ 0.067 & 1.458 $\pm$ 0.068 \\ 
			& & Te & -0.081 $\pm$ 0.011 & -0.081 $\pm$ 0.011 & -0.091 $\pm$ 0.008 & -0.089 $\pm$ 0.011 & -0.089 $\pm$ 0.011 \\ 
			& & Total & 58.209 & 58.209 & 58.711 & 58.922 & 58.926 \\ 
			\midrule
			& Total Energy (eV) & & -566.454 & -566.454 & -567.718 & -567.846 & -567.847 \\ 		     
			\bottomrule
			\bottomrule
			\label{tab:4.14}
		\end{tabular}
	}
\end{table}

For the concentrations \( x = 0.25, 0.50, \) and \( 0.75 \), we obtain formation energies of -0.163, -0.314, and -0.471 eV/f.u., respectively, along with mixing energies of 0.054, 0.042, and 0.025 eV/f.u. These negative formation energies indicate that the random alloy exhibits stable behavior at the studied concentrations. The positive mixing energies suggest that the \( 4 \times 3 \times 1 \) supercell contains heterogeneous interaction domains among the constituent atoms, consistent with the characteristics of a truly random alloy.

We present the projected density of states for this random alloy in Fig. \ref{fig:dosplotCrFealloys}.

\begin{figure}[H]
	\centering
	\begin{subfigure}{.50\textwidth}
		\centering
		\includegraphics[width=.95\linewidth]{Figures/dos_alloyCrFe25.png}
	\end{subfigure}%
	\hfill % Fill space between subfigures
	\begin{subfigure}{.50\textwidth}
		\centering
		\includegraphics[width=.95\linewidth]{Figures/dos_alloyCrFe50.png}
	\end{subfigure}
	
	\begin{subfigure}{.50\textwidth}
		\centering
		\includegraphics[width=.95\linewidth]{Figures/dos_alloyCrFe75.png}
	\end{subfigure}%
	\caption{Projected density of states for the Cr$_{1-x}$GeFe$_{x}$Te$_{3}$ random alloys at concentrations (a) \( x = 0.25 \), (b) \( x = 0.50 \), and (c) \( x = 0.75 \).}
	\label{fig:dosplotCrFealloys}
\end{figure}

As illustrated in Fig. \ref{fig:dosplotCrFealloys} \textbf{(a)}, at a concentration of \( x = 0.25 \), well-behaved localized \( d \) states are observed around the Fermi level for Fe atoms in the spin-down channel, while conduction and valence bands in the spin-up channel are less localized. Additionally, the \( d \) states of Cr atoms are hybridized with the \( p \) states of Te in both spin channels.

For the concentration of \( x = 0.50 \) shown in Fig. \ref{fig:dosplotCrFealloys} \textbf{(b)}, we again observe localized \( d \) states for Fe atoms around the Fermi level. However, the \( d \) states of Cr are not localized within the valence or conduction bands, as both are hybridized with the \( p \) states of Te.

Conversely, for the concentration of \( x = 0.75 \) presented in Fig. \ref{fig:dosplotCrFealloys} \textbf{(c)}, the \( d \) states of Fe atoms contribute minimally to the density of states of the alloy. In contrast, the \( d \) states of Cr are better defined, particularly with respect to the \( e_g \) orbitals, which exhibit more localization compared to the \( t_{2g} \) orbitals. It is also noteworthy that there exists a strong hybridization, similar to the other concentrations, between the \( d \) states of Cr and Fe and the \( p \) states of Te.


\subsection{\texorpdfstring{Fe$_{1-x}$GeMn$_{x}$Te$_{3}$ random alloys}{Fe1-xGeMnxTe3 random alloys}}
Finally, we present the results for the random alloy incorporating Fe and Mn atoms. Table \ref{tab:4.15} displays the mean magnetic moments of the atoms involved in the random alloy Fe$_{1-x}$GeMn$_{x}$Te$_{3}$ and their corresponding total energy for each concentration. Generally, no significant errors are associated with these results. We found formation energies of -0.384, -0.304, and -0.169 eV/f.u. for concentrations of x = 0.25, 0.50, and 0.75, respectively. The mixing energies for the same concentrations are -0.011, -0.030, and 0.006 eV/f.u., indicating that these alloys are thermodynamically stable. However, at x = 0.75, we observe an inhomogeneous alloy, as its positive mixing energy leads to structural tensions within the alloy.

\begin{table}[H]
	\centering
	\sisetup{table-format=3.0, table-number-alignment=center, group-separator={,}}
	\setlength{\extrarowheight}{0.5ex}
	\caption{Average magnetic moments per atom and total energies of the random alloys $Fe_{1-x}GeMn_{x}Te_{3}$ at the concentrations $x=0.25$, $0.50$, and $0.75$ during each stage of the implemented relaxation cycle. Note that the standard deviation values are minimal, ensuring realistic calculations.}
	\resizebox{15cm}{!}{
		\begin{tabular}{cccccccc}
			\toprule
			\toprule
			\cellcolor{WhiteSmoke!70!Lavender} & \cellcolor{WhiteSmoke!70!Lavender} &  \cellcolor{WhiteSmoke!70!Lavender} & \multicolumn{5}{c}{\cellcolor{WhiteSmoke!70!Lavender}Relaxation}\\ 
			\cline{4-8}
			\multirow{-2}{*}{\cellcolor{WhiteSmoke!70!Lavender}Concentration $x \%$} & \multirow{-2}{*}{\cellcolor{WhiteSmoke!70!Lavender}Property} & \multirow{-2}{*}{\cellcolor{WhiteSmoke!70!Lavender}Atom} & \cellcolor{WhiteSmoke!70!Lavender}Step 1 & \cellcolor{WhiteSmoke!70!Lavender}Step 2 & \cellcolor{WhiteSmoke!70!Lavender}Step 3 & \cellcolor{WhiteSmoke!70!Lavender}Step 4 & \cellcolor{WhiteSmoke!70!Lavender}Step 5 \\
			\midrule
			\midrule
			\multirow{5}{*}{25} & & Fe & -0.038 $\pm$ 2.875 & -0.415 $\pm$ 1.074 & -0.938 $\pm$ 0.077 & -0.795 $\pm$ 0.472 & -0.625 $\pm$ 0.884 \\
			& Magnetic & Ge & 0.017 $\pm$ 0.004 & 0.016 $\pm$ 0.003 & 0.015 $\pm$ 0.004 & 0.015 $\pm$ 0.004 & 0.015 $\pm$ 0.004 \\ 
			& moment ($\mu_B /atom$) & Mn & 3.539 $\pm$ 0.016 & 3.496 $\pm$ 0.013 & 3.443 $\pm$ 0.066 & 3.444 $\pm$ 0.067 & 3.458 $\pm$ 0.066 \\ 
			& & Te & -0.050 $\pm$ 0.017 & -0.050 $\pm$ 0.017 & -0.045 $\pm$ 0.025 & -0.046 $\pm$ 0.024 & -0.046 $\pm$ 0.024 \\ 
			& & Total & 60.266 & 57.230 & 53.479 & 54.257 & 55.494 \\ 
			\midrule
			& Total Energy (eV) & & -553.072 & -552.183 & -554.872 & -554.850 & -554.864 \\ 		                                           
			\midrule
			\midrule
			\multirow{5}{*}{50} & & Fe & 2.427 $\pm$ 0.018 & 2.427 $\pm$ 0.018 & 0.828 $\pm$ 0.864 & 0.797 $\pm$ 0.583 & 0.879 $\pm$ 0.883 \\
			& Magnetic & Ge & 0.008 $\pm$ 0.004 & 0.008 $\pm$ 0.004 & 0.012 $\pm$ 0.003 & 0.011 $\pm$ 0.003 & 0.011 $\pm$ 0.003 \\ 
			& moment ($\mu_B /atom$) & Mn & 3.509 $\pm$ 0.019 & 3.510 $\pm$ 0.019 & 3.457 $\pm$ 0.090 & 3.438 $\pm$ 0.091 & 3.430 $\pm$ 0.088 \\ 
			& & Te & -0.056 $\pm$ 0.006 & -0.056 $\pm$ 0.006 & -0.052 $\pm$ 0.013 & -0.052 $\pm$ 0.013 & -0.052 $\pm$ 0.013 \\ 
			& & Total & 67.366 & 67.375 & 47.952 & 47.361 & 48.173 \\ 
			\midrule
			& Total Energy (eV) & & -544.153 & -544.156 & -547.343 & -547.373 & -548.038 \\ 		     
			\midrule
			\midrule
			\multirow{5}{*}{75} & & Fe & 2.306 $\pm$ 0.011 & 2.308 $\pm$ 0.011 & 1.042 $\pm$ 0.180 & 1.033 $\pm$ 0.180 & 1.035 $\pm$ 0.179 \\
			& Magnetic & Ge & 0.008 $\pm$ 0.003 & 0.008 $\pm$ 0.003 & 0.018 $\pm$ 0.003 & 0.025 $\pm$ 0.012 & 0.025 $\pm$ 0.012 \\ 
			& moment ($\mu_B /atom$) & Mn & 3.447 $\pm$ 0.015 & 3.448 $\pm$ 0.015 & 3.297 $\pm$ 0.031 & 3.288 $\pm$ 0.031 & 3.288 $\pm$ 0.031 \\ 
			& & Te & -0.057 $\pm$ 0.005 & -0.057 $\pm$ 0.005 & -0.049 $\pm$ 0.013 & -0.049 $\pm$ 0.013 & -0.049 $\pm$ 0.013 \\ 
			& & Total & 58.096 & 58.095 & 35.116 & 34.924 & 34.965 \\ 
			\midrule
			& Total Energy (eV) & & -535.253 & -535.254 & -539.912 & -539.914 & -539.909 \\ 		     
			\bottomrule
			\bottomrule
			\label{tab:4.15}
		\end{tabular}
	}
\end{table}

As with the other random alloys, we present the projected density of states (PDOS) for the different concentrations considered, as shown in Fig. \ref{fig:dosplotFeMnalloys}.

\begin{figure}[H]
	\centering
	\begin{subfigure}{.50\textwidth}
		\centering
		\includegraphics[width=.95\linewidth]{Figures/dos_alloyFeMn25.png}
	\end{subfigure}%
	\hfill % Fill space between subfigures
	\begin{subfigure}{.50\textwidth}
		\centering
		\includegraphics[width=.95\linewidth]{Figures/dos_alloyFeMn50.png}
	\end{subfigure}
	\begin{subfigure}{.60\textwidth}
		\centering
		\includegraphics[width=\linewidth]{Figures/dos_alloyFeMn75.png}
	\end{subfigure}
	\caption{Projected density of states of the random alloys Fe$_{1-x}$GeMn$_{x}$Te$_{3}$ for ($\mathbf{a}$) at $x=0.25$, for ($\mathbf{b}$) at $x=0.50$, and for ($\mathbf{c}$) at $x=0.75$. In all cases, a strong hybridization of the $d$ orbitals of magnetic atoms with the $p$ orbitals of Te is observed.}
	\label{fig:dosplotFeMnalloys}
\end{figure}

As shown in Fig. \ref{fig:dosplotFeMnalloys}, the concentration of Fe atoms increases the presence of $d$ orbitals in the projected density of states. 

In Fig. \ref{fig:dosplotFeMnalloys}\textbf{(a)}, we observe bound states for the $t_{2g}$ orbitals in the range of -4 to -2 eV for the spin-up channel, while bound states for the same orbitals are present in the range of 0 to 2 eV for the spin-down channel. The $e_{g}$ orbitals appear to be more accurately described, as they are slightly more localized, particularly for the conduction band located around 0.95 eV above the Fermi level. In contrast, the contribution of Fe to the PDOS is weaker, and the $e_{g}$ orbitals are better localized compared to their $t_{2g}$ counterparts. The $d$ states of both Fe and Mn atoms exhibit strong hybridization with the $p$ states of Te.  

In Fig. \ref{fig:dosplotFeMnalloys}\textbf{(b)}, the $t_{2g}$ orbitals of Fe appear delocalized, while the localized $e_{g}$ orbitals of the same species are better described. Additionally, bound states for the $d$ states of Mn in the spin-up channel are observed in the valence bands around -4 to -2 eV, with similar states present in the conduction bands from 0 to 2 eV.

Finally, in Fig. \ref{fig:dosplotFeMnalloys}\textbf{(c)}, the delocalized $t_{2g}$ orbitals are more pronounced compared to the concentration $x=0.50$. Similarly, the $e_{g}$ orbitals for Fe are better localized than in the previous two concentrations. Furthermore, it is evident that the $d$ orbitals of Mn are completely hybridized with the $p$ states of Te.
