% Chapter Template

\chapter{Theoretical Background} % Main chapter title

\label{Chapter2} % Change X to a consecutive number; for referencing this chapter elsewhere, use \ref{ChapterX}

\lhead{Chapter 2. \emph{Theoretical Background}}
This chapter is devoted to present the theoretical foundations and formalism of Density Functional Theory (DFT) and related methods required for the development of the results presented in this work. Starting with the many-body Schrödinger equation, this chapter covers the Born-Oppenheimer approximation, the Hartree-Fock approximation, Hohenberg-Kohn theorems, the Kohn-Sham equations, exchange-correlation functionals, and definitions on Ab initio molecular dynamics (AIMD) and machine learning force fields (MLFFs), along with implementation details in the Vienna Ab initio Simulation Package (VASP).

Ultimately, the aim of this chapter is to provide a comprehensive understanding of the theoretical framework that underpins the computational methods utilised in this work, enabling the reader to grasp the principles and assumptions that govern the simulations and analyses performed throughout.

\section{Many Body Schrödinger Equation}
In the realm of materials science, comprehending particle behaviour within a system inevitably demands resorting to intricate principles of quantum mechanics. Our journey shall commence by describing the physical laws that shape the interactions among particles constituting a system---electrons and nuclei alike. 

\subsection{The Coulomb Interaction}

Materials may be thought of as complex assemblies of electrons and nuclei, held together by a delicate balance between attractive Coulomb interactions---primarily between electrons and nuclei---and repulsive interactions between like-charged particles, such as electron-electron and nucleus-nucleus pairs, which govern the overall dynamics of the material system\supercite{giustino2014materials, sholl2023density, kaxiras2003atomic}.  From classical electrostatics, these interactions can be mathematically expressed as follows: 

\begin{itemize}
\item Electron-electron interactions 
  \begin{equation}
    \label{eq1}
    \hat{V}_{ee} = \frac{1}{2} \sum_{i\neq j} \frac{e^2}{4\pi\epsilon_0 |\mathbf{r}_i - \mathbf{r}_j|}
  \end{equation}
\item Electron-nucleus interactions 
  \begin{equation}
    \label{eq2}
    \hat{V}_{nn} = \frac{1}{2} \sum_{I\neq J} \frac{e^2}{4\pi\epsilon_0} \frac{Z_I Z_J}{|\mathbf{R}_I - \mathbf{R}_J|}
  \end{equation}
\item Electron-nuclei interactions 
\begin{equation}
    \label{eq3}
    \hat{V}_{en} = -\sum_{i\neq I} \frac{e^2}{4\pi\epsilon_0} \frac{Z_I}{|\mathbf{r}_i - \mathbf{R}_I|}
  \end{equation}
\end{itemize}

where $e$ is the electronic charge, $\epsilon_0$ is the vacuum permittivity, $Z_I$ and $Z_J$ are the atomic numbers of nuclei $I$ and $J$, respectively, and $\mathbf{r}_i$ and $\mathbf{R}_I$ are the position vectors of electrons and nuclei, respectively. Moreover, we must also consider the kinetic energy of the collection of electrons and nuclei
\begin{equation}
    \label{eq4}
    \hat{T} = -\sum_i \frac{\hbar^2}{2m_e} \nabla_i^2 - \sum_I \frac{\hbar^2}{2M_I} \nabla_I^2
\end{equation}
where $\hbar$ is the reduced Planck's constant, $m_e$ is the electron mass, and $M_I$ is the mass of the nucleus $I$.

\subsection{The Time-Independent Schrödinger Equation}
The Time-Independent Schrödinger Equation (TISE) lies at the heart of non-relativistic quantum mechanics, providing a mathematical framework to describe stationary electronic states of quantum systems. It takes the following form:
\begin{equation}
  \label{eq5}
  \hat{H} \psi(\mathbf{r}) = E \psi(\mathbf{r})
\end{equation}
where $\hat{H}$ is the Hamiltonian of the system, encompassing both kinetic and potential energies, $\psi(\mathbf{r})$ is the wavefunction of the system, and $E$ is the energy eigenvalue associated with the state described by $\psi(\mathbf{r})$.  It is important to note that Equation \ref{eq5} is applicable to a single particle, yet a material system is composed of many electrons (N) and nuclei (M) with spatial coordinates $\mathbf{r}_1, \mathbf{r}_2, \ldots, \mathbf{r}_N$ and $\mathbf{R}_1, \mathbf{R}_2, \ldots, \mathbf{R}_M$, respectively. Therefore, we must introduce a so called many-body wavefunction given by:
\begin{equation}
  \Psi(\mathbf{r}_1, \mathbf{r}_2, \ldots, \mathbf{r}_N, \mathbf{R}_1, \mathbf{R}_2, \ldots, \mathbf{R}_M)
  \label{eq5b}
\end{equation}
On this basis, the many-body version of Equation \ref{eq5} shall be constructed by combining the kinetic (Equation \ref{eq4}) and potential (Equations \ref{eq1}, \ref{eq2}, and \ref{eq3}) energy contributions, leading to the following expression:
\begin{equation}
  \label{eq6}
  \begin{split}
    \left[
      -\sum_i \frac{\hbar^2}{2m_e} \nabla_i^2 
      - \sum_I \frac{\hbar^2}{2M_I} \nabla_I^2 
      + \frac{1}{2} \sum_{i\neq j} \frac{e^2}{4\pi\epsilon_0 |\mathbf{r}_i - \mathbf{r}_j|} +\right. \\
      \left. \frac{1}{2} \sum_{I\neq J} \frac{e^2}{4\pi\epsilon_0} \frac{Z_I Z_J}{|\mathbf{R}_I - \mathbf{R}_J|} 
      - \sum_{i, I} \frac{e^2}{4\pi\epsilon_0} \frac{Z_I}{|\mathbf{r}_i - \mathbf{R}_I|}
    \right]\Psi = E_{\text{tot}} \Psi
  \end{split}
\end{equation}
Equation \ref{eq6} provides a complete description of the stationary states of a non-relativistic many-body system, under time-independent conditions and in the absence of external fields. Additionally, we can achieve a more compact formulation by introducing the concept of atomic units. To this end, let us consider the simplest electron-nucleus  system---the hydrogen atom---where the electron orbital has an average radius $a_0 \approx 0.529 \, \text{Å}$. Thereby, the Coulomb energy for such a system is given by:
\begin{equation}
  \label{eq7}
  E_{\text{Ha}} = \frac{e^2}{4 \pi \epsilon_0 a_0}
\end{equation}
where 'Ha' stands for Hartree. Within this framework, the Hartree energy represents the Coulomb interaction between two fundamental charges separated by a distance of one Bohr radius ($a_0$). Moreover, the angular momentum quantisation condition for the electron in the hydrogen atom is given by
\begin{equation}
  \label{eq8}
  m_e v a_0 = \hbar
\end{equation}
Additionally, the equilibrium condition between the nuclear attraction and the electron's centrifugal force can be expressed as:
\begin{equation}
  \label{eq9}
  \frac{e^2}{4 \pi \epsilon_0 a_0^2} = \frac{m_e v^2}{a_0}
\end{equation}
By combining Equations \ref{eq7}, \ref{eq8}, and \ref{eq9}, we derive the 
following relationships:
\begin{equation}
  \label{eq10}
  \frac{e^2}{4 \pi \epsilon_0 a_0} = \frac{\hbar^2}{m_e a_0^2}
\end{equation}
\begin{equation}
  \label{eq11}
  \frac{1}{2} m_e v^2 = \frac{1}{2}E_{\text{Ha}}
\end{equation}
This latter relation showcases that the kinetic energy is of the same order as $E_\text{Ha}$, rendering it convenient to normalise Equation \ref{eq6} by this quantity:
\begin{equation}
  \label{eq12}
  \begin{split}
    \left[
      -\sum_i \frac{1}{2}a_0^2 \nabla_i^2 
      - \sum_I \frac{1}{2} \frac{1}{(M_I/m_e)} \nabla_I^2 
      + \frac{1}{2} \sum_{i\neq j} \frac{a_0}{|\mathbf{r}_i - \mathbf{r}_j|}\right. \\
      \left. +  \frac{1}{2} \sum_{I\neq J} Z_I Z_J \frac{a_0}{|\mathbf{R}_I - \mathbf{R}_J|} 
      - \sum_{i, I} Z_I \frac{a_0}{|\mathbf{r}_i - \mathbf{R}_I|}
    \right]\Psi = \frac{E_{\text{tot}}}{E_{\text{Ha}}} \Psi
  \end{split}
\end{equation}
A final simplification involves setting our energy units to Ha, distance units to $a_0$, and mass units to $m_e$. The last missing constant $e$ is set to 1, leading to the following expression: 
\begin{equation}
  \label{eq13}
  \begin{split}
    \left[
      -\sum_i \frac{\nabla_i^2}{2}
      - \sum_I \frac{\nabla_I^2}{2 M_I} 
      + \frac{1}{2} \sum_{i\neq j} \frac{1}{|\mathbf{r}_i - \mathbf{r}_j|} + \right. \\
      \left. \frac{1}{2} \sum_{I\neq J} \frac{Z_I Z_J} {|\mathbf{R}_I - \mathbf{R}_J|} 
      - \sum_{i, I} \frac{Z_I}{|\mathbf{r}_i - \mathbf{R}_I|}
    \right]\Psi = E_{\text{tot}} \Psi
  \end{split}
\end{equation}

Ultimately, even though Equation \ref{eq13} provides an exact method capable of yielding various properties of a material system---such as elastic, thermal, and electronic properties---a combination of mathematical complexity and computational limitations renders it intractable to solve for any realistic system. Moreover, the wavefunction contains vastly more information than is necessary to describe most observable properties of a material. Therefore, we must resort to alternative formulations that allow us to extract only the relevant information from the wavefunction whilst reducing the computational cost of the calculations. The remainder of this chapter is dedicated to presenting such alternative approaches that ultimately lead to the computational methods employed throughout this work.

%-----------------------------------
%	SUBSECTION 1
%-----------------------------------
\section{The Born-Oppenheimer Approximation}
For atoms in a solid, we can think of nuclei as being held immobile in a fixed position, while electrons instantaneously react to any nuclei movement. This assumption is based on the fact that nuclei are much heavier than electrons---by three to four orders of magnitude---making the former behave like classical particles. Thereby, we can rewrite the many-body wavefunction as a product of two wavefunctions:
\begin{equation}
  \Psi(\mathbf{r}_1, \mathbf{r}_2, \ldots, \mathbf{r}_N, \mathbf{R}_1, \mathbf{R}_2, \ldots, \mathbf{R}_M) = \psi_{\mathbf{R}}(\mathbf{r}_1, \mathbf{r}_2, \ldots, \mathbf{r}_N)\chi(\mathbf{R})
  \label{eq14}
\end{equation}
where $\psi_{\mathbf{R}}$ is the electronic wavefunction parametrised by the nuclear positions $\mathbf{R}$, and $\chi$ is the nuclear wavefunction. Furthermore, this significant mass disparity enables a systematic approximation scheme, in which the electronic wavefunction is solved for fixed nuclei, and its solution is used as an effective potential for the nuclear dynamics afterwards. First, nuclei kinetic energy is neglected, as their positions are assumed to be fixed: 
\begin{equation}
  \label{eq15}
  \sum_I \frac{\nabla_I^2}{2 M_I} = 0
  \quad \text{and} \quad  E = E_{\text{tot}} - \sum_{I<J} \frac{Z_I Z_J}{|\mathbf{R}_I - \mathbf{R}_J|}
\end{equation}
Following, we define the Coulomb potential of the nuclei experienced by the electrons as:
\begin{equation}
  \label{eq16}
  V_{\text{n}}(\mathbf{r}) = - \sum_{I} \frac{Z_I}{|\mathbf{r} - \mathbf{R}_I|}
\end{equation}
Then, Equation \ref{eq13} can be rewritten as:
\begin{equation}
  \label{eq17}
  \left[
    -\sum_i \frac{\nabla_i^2}{2} + \sum_i V_{\text{n}}(\mathbf{r}_i) + \frac{1}{2} \sum_{i\neq j} \frac{1}{|\mathbf{r}_i - \mathbf{r}_j|} 
  \right] \Psi = E \Psi 
\end{equation}
Finally, by using Equation \ref{eq14}, we can define the electronic and nuclear Schrödinger equations as follows:
\begin{equation}
  \label{eq18}
  \left[
    -\sum_i \frac{\nabla_i^2}{2} + \sum_i V_{\text{n}}(\mathbf{r}_i) + \frac{1}{2} \sum_{i\neq j} \frac{1}{|\mathbf{r}_i - \mathbf{r}_j|} 
  \right] \psi_{\mathbf{R}} = E_{\mathbf{R}} \psi_{\mathbf{R}}
\end{equation}
\begin{equation}
  \label{eq19}
  \left[
    -\sum_I \frac{\nabla_I^2}{2 M_I} + \sum_{I<J} \frac{Z_I Z_J}{|\mathbf{R}_I - \mathbf{R}_J|} + E(\mathbf{R}_1,\dots,\mathbf{R}_M)\right] \chi(\mathbf{R}) = E_{\text{tot}} \chi(\mathbf{R})
\end{equation}
where $E_{\mathbf{R}}= E(\mathbf{R}_1,\dots,\mathbf{R}_M)$ is the electronic surface energy, which is a function of the nuclear positions, and serves as an effective potential shaping the nuclear dynamics. 

%-----------------------------------
%	SUBSECTION 2
%-----------------------------------

\section{Hartree-Fock Approximation}
The essence of the Hartree-Fock approximation (HFA) is approximate the interacting many-electron system (Equation \ref{eq17}) by a set of non-interacting single-particle problems subject to an effective mean-field potential\supercite{martin2016interacting, helgaker2014molecular, feng2005introduction}. As a means to this, we first rewrite the total wavefunction for a system with N electrons as the product of single-electron wavefunctions---often referred to as the Hartree approximation\supercite{Hartree1928}---as showcased in Equation \ref{eq20}. 
\begin{equation}
  \Psi^H(\mathbf{r}_1, \dots, \mathbf{r}_N) = \prod_{i=1}^N \phi_i(\mathbf{r}_i)
  \label{eq20}
\end{equation}
Following, we construct the total energy functional as the expectation value of the Hamiltonian operator:
\begin{equation}
  \label{eq21}
\begin{aligned}
  E^{H}[\{\phi_i\}] &= \bigg\langle \Psi^H \bigg| \hat{H} \bigg| \Psi^H \bigg\rangle 
\end{aligned}
\end{equation}
Expanding Equation \ref{eq21}:
\begin{equation}
  \label{eq22}
  \begin{split}
    E^{H}[\{\phi_i\}] &= \sum^N_{i=1} \int \phi_i^*(\mathbf{r}) \left(-\frac{\nabla^2}{2} + V_{\text{n}}(\mathbf{r})\right) \phi_i(\mathbf{r}) d\mathbf{r}  + \frac{1}{2} \sum_{i\neq j} \int\int \frac{|\phi_i(\mathbf{r})|^2 |\phi_j(\mathbf{r'})|^2 }{|\mathbf{r} - \mathbf{r'}|} d\mathbf{r} d\mathbf{r'}
  \end{split}
\end{equation}
where the first term adds up the kinetic energy and electron-nuclear attraction overall electrons, whilst the second accounts for the classical electron-electron repulsion energy averaged over the electron density distribution. In order to find the set of orbitals $\{\phi_i\}$ that minimises the total energy functional we use the variational principle where we shall impose the orthonormality condition: 
\begin{equation}
  \label{eq23}
  \int \phi_i^*(\mathbf{r}) \phi_j(\mathbf{r}) d\mathbf{r} = \delta_{ij}
\end{equation}
for what we introduce the Lagrange multipliers $\lambda_{ij}$ to enforce these constraints and define the Lagrangian:
\begin{equation}
  \label{eq24}
  \mathcal{L} = E^H[\{\phi_i\}] - \sum_{i=1}^{N}\lambda_{ij} ( \langle\phi_i|\phi_j\rangle - \delta_{ij})
\end{equation}
which ultimately simplifies to:
\begin{equation}
  \label{eq25}
  \mathcal{L} = E^H[\{\phi_i\}] - \sum_{i=1}^{N}\varepsilon_{i} ( \langle\phi_i|\phi_i\rangle - 1)
\end{equation}

Then, we need to compute the derivative of $\mathcal{L}$ with respect to $\phi_i^*$ and set it to zero:
\begin{equation}
  \label{eq26}
  \frac{\delta \mathcal{L}}{\delta \phi_i^*}(\mathbf{r}) = 0
\end{equation}
which yield to the Hartree equation:
\begin{equation}
  \label{eq27}
  \left[-\frac{\nabla^2}{2} + V_{\text{n}}(\mathbf{r}) + V^H_i(\mathbf{r})\right]\phi_i(\mathbf{r})  = \varepsilon_i \phi_i(\mathbf{r})
\end{equation}
where $V_n(\mathbf{r})$ represents the electrostatic interaction between electrons and nuclei, and the Hartree potential 
\begin{equation}
  \label{eq28}
  V^H_i(\mathbf{r}) = \sum_{j\neq i} \int \frac{|\phi_j(\mathbf{r'})|^2}{|\mathbf{r} - \mathbf{r'}|} d\mathbf{r'}
\end{equation}
accounts for the average electrostatic interaction experienced by the $i$-th electron due to all other electrons in the system. This effective mean-field potential replaces the electron-electron interactions, effectively simplifying the many-body problem into single-particle problems. 
\subsection{The Pauli Exclusion Principle}
So far, we have introduced the Hartree approximation, which assumes that the many-electron wavefunction can be expressed as a product of single-particle wavefunctions. However, this approach does not account for the indistinguishability of electrons and the Pauli exclusion principle, which states that no two fermions---half spin particles, such as electrons---can reside in the same quantum state simultaneously. In doing so, it imposes a restriction on the possible configurations of electrons in a system that shall be accounted for.
 
 In order to achieve this, V. Fock\supercite{Fock1930} introduced a different approximation to the wavefunction by using a Slater determinant---a mathematical construct that combines one-electron wavefunctions in such a way that satisfies the antisymmetry principle. This is done by expressing the overall wavefunction as the determinant of a matrix of single-electron wavefunctions:
\begin{equation}
  \Psi^{HF}(\mathbf{r}_1, \dots, \mathbf{r}_N) = \frac{1}{\sqrt{N!}} \begin{vmatrix}
    \phi_1(\mathbf{r}_1) & \phi_1(\mathbf{r}_2) & \dots & \phi_1(\mathbf{r}_N)\\
    \phi_2(\mathbf{r}_1) & \phi_2(\mathbf{r}_2) & \dots & \phi_2(\mathbf{r}_N)\\
    \vdots & \vdots & \ddots & \vdots\\
    \phi_N(\mathbf{r}_1) & \phi_N(\mathbf{r}_2) & \dots & \phi_N(\mathbf{r}_N)
  \end{vmatrix}
  \label{eq29}
\end{equation}
where $1/\sqrt{N!}$ is a normalisation factor. To illustrate this, consider a two-electron system with single-particle wavefunctions $\phi_1(\mathbf{r})$ and $\phi_2(\mathbf{r})$. The Slater determinant for this system would be:
\begin{equation}
  \Psi^{HF}(\mathbf{r}_1, \mathbf{r}_2) = \frac{1}{\sqrt{2}} \begin{vmatrix}
    \phi_1(\mathbf{r}_1) & \phi_1(\mathbf{r}_2)\\
    \phi_2(\mathbf{r}_1) & \phi_2(\mathbf{r}_2)
  \end{vmatrix} = \frac{1}{\sqrt{2}} \left[\phi_1(\mathbf{r}_1)\phi_2(\mathbf{r}_2) - \phi_1(\mathbf{r}_2)\phi_2(\mathbf{r}_1))\right]
  \label{eq30}
\end{equation}
Evidently, $\Psi^{HF}(\mathbf{r}_1, \mathbf{r}_2) = -\Psi^{HF}(\mathbf{r}_2, \mathbf{r}_1)$, which satisfies the antisymmetr principle. 
\subsection{The Hartree-Fock Equations}
The Hartree-Fock equations are derived in a similar manner we addressed the Hartree equations. We first define the total energy with the Hartree-Fock wavefunction (Equation \ref{eq29}) 
\begin{equation}
  \label{eq31}
  \begin{split}
    E^{HF}[\{\phi_i\}] &= \bigg\langle \Psi^{HF} \bigg| \hat{H} \bigg| \Psi^{HF} \bigg\rangle\\
    &= \sum_{i} \langle \phi_i \big|\frac{\nabla^2}{2} + V_{\text{n}}(\mathbf{r})\big| \phi_i \rangle\\
    &+ \frac{1}{2} \sum_{i\neq j} \langle \phi_i \phi_j \big| \frac{1}{|\mathbf{r}_i - \mathbf{r}_j|} \big| \phi_i \phi_j\rangle\\
    &- \frac{1}{2} \sum_{i\neq j} \langle \phi_i \phi_j \big| \frac{1}{|\mathbf{r}_i - \mathbf{r}_j|} \big| \phi_j \phi_i\rangle
  \end{split}
\end{equation}
Consecutively, using the variational principle, we derive the Hartree-Fock equations: 
\begin{equation}
  \label{eq32}
  \left[-\frac{\nabla^2}{2} + V_{\text{n}}(\mathbf{r}) + V^H_i(\mathbf{r}) +\right]\phi_i(\mathbf{r})
  - \sum_{j\neq i} \langle \phi_j \big| \frac{1}{|\mathbf{r}_i - \mathbf{r}_j|} \big| \phi_i \rangle \phi_j(\mathbf{r})
  = \varepsilon_i \phi_i(\mathbf{r})
\end{equation}
Noticeably, Equation \ref{eq32} has an extra term compared with the Hartree equation (Equation \ref{eq27}). This term is called the "exchange" term\supercite{kaxiras2003atomic}, and describes the effects of exchange between electrons. It is convenient to try to express the Hartree-Fock equations in a more compact form, so we define the single-particle and total densities as 
\begin{equation}
  \label{eq33}
  \rho_i(\mathbf{r}) = |\phi_i(\mathbf{r})|^2
\end{equation}
\begin{equation}
  \label{eq34}
  \rho(\mathbf{r}) = \sum_i \rho_i(\mathbf{r}) 
\end{equation}
so the Hartree potential can be expressed as 
\begin{equation}
  \label{eq35}
  V^H_i(\mathbf{r}) = \sum_{j\neq i}  \int \frac{\rho_j(\mathbf{r'})}{|\mathbf{r} - \mathbf{r'}|} d\mathbf{r'} 
  = \int \frac{\rho(\mathbf{r'}) - \rho_i(\mathbf{r'})}{|\mathbf{r} - \mathbf{r'}|} d\mathbf{r'}
\end{equation}
Therefore, the single-particle exchange density can be constructed as 
\begin{equation}
  \label{eq36}
  \rho^X_i(\mathbf{r}, \mathbf{r'}) = \sum_{j\neq i}\frac{\phi_i(\mathbf{r'})\phi^*_i(\mathbf{r})\phi_j(\mathbf{r})\phi^*_j(\mathbf{r'})}{\phi_i(\mathbf{r})\phi^*_i(\mathbf{r})}
\end{equation}
Finally, the Hartree-Fock equations take the form 
\begin{equation}
  \label{eq37}
  \left[-\frac{\nabla^2}{2} + V_{\text{n}}(\mathbf{r}) + V^H_i(\mathbf{r}) + V^X_i(\mathbf{r})\right]\phi_i(\mathbf{r}) = \varepsilon_i \phi_i(\mathbf{r})
\end{equation}
where $V^X_i$ stands for the exchange potential
\begin{equation}
  \label{eq38}
  V^X_i(\mathbf{r}) = -\int \frac{\rho^X_i(\mathbf{r}, \mathbf{r'})}{|\mathbf{r} - \mathbf{r'}|} d\mathbf{r'}
\end{equation}

%----------------------------------------------------------------------------------------
%	SECTION 2
%----------------------------------------------------------------------------------------

\section{Density Functional Theory}

Hello 


\subsection{First Hohenberg-Kohn Theorem}

\subsection{Second Hohenberg-Kohn Theorem}

\subsection{Kohn-Sham Equations}

\subsection{Exchange-Correlation Functionals}

\section{Ab initio Molecular Dynamics}

\subsection{Born-Oppenheimer Molecular Dynamics}

\subsection{Car-Parrinello Molecular Dynamics}

\section{Machine Learning Force Fields}

\section{Computational Implementation in VASP}
\subsection{Plane Wave Basis Set}
\subsection{K-Point Sampling and Cutoff Energy}
\subsection{Pseudopotentials}
\section{Density of States}



