%----------------------------------------------------------------------------------------

% Chapter 3

\chapter{Methodology} % Main chapter title

\label{Chapter3} % For referencing the chapter elsewhere, use \ref{Chapter1} 

\lhead{Chapter 3. \emph{Methodology}} % This is for the header on each page - perhaps a shortened title



The computational investigations are conducted using the Vienna Ab initio Simulation Package (VASP). The crystal structures analyzed conform to the general formula XGeTe$_{3}$, which includes three specific compounds:  

\begin{itemize}  
	\item CrGeTe$_{3}$, characterized by a chromium electronic configuration of \([Ar]: 3d^5 4s^1\),  
	\item MnGeTe$_{3}$, with a manganese electronic configuration of \([Ar]: 3d^6 4s^2\), and  
	\item FeGeTe$_{3}$, exhibiting an iron electronic configuration of \([Ar]: 3d^7 4s^2\).  
\end{itemize}  

Each crystal structure is defined by a unit cell containing 10 atoms, corresponding to two formula units (f.u.) per unit cell. The common elements in these structures, germanium (Ge) and tellurium (Te), have electronic configurations of \([Ar]: 4s^2 4p^2\) and \([Ar]: 5s^2 5p^4\), respectively. Additionally, these structures share identical symmetry characteristics, corresponding to space group 147.  

The electronic properties are analyzed using Projector Augmented Wave (PAW) potentials. The ab initio study begins with spin-polarized calculations on the CrGeTe$_{3}$ monolayer, which are subsequently extended to the MnGeTe$_{3}$ and FeGeTe$_{3}$ monolayers. The Perdew-Burke-Ernzerhof (PBE) functional is initially employed. Since the ground state of each monolayer may involve magnetic configurations (either ferromagnetic or antiferromagnetic), we systematically determine the appropriate cutoff energy and k-point mesh for each structure.  

A suitable energy cutoff for all three monolayers is established, adhering to the convergence criterion outlined in Section \ref{cutoff-energy}. A plane-wave basis with a cutoff energy of 500 eV is applied, meeting a convergence criterion of less than 1 meV/f.u., achieved through a Monkhorst-Pack mesh. We optimize the k-point mesh by performing several ionic position optimizations to derive the optimal lattice parameters, as discussed in the Two-Dimensional Equation of State (Section \ref{2D.eqs}). A convergence criterion of 1 m\AA. is adopted, leading to the selection of a \(10 \times 10 \times 1\) k-point grid.  

The calculations incorporate a self-consistency loop break condition of \(1 \times 10^{-6}\) eV to minimize undesired Pulay stress and ensure an adequate plane-wave basis set. This two-step convergence criterion ensures accurate description of any XGeTe$_{3}$ crystal structure, regardless of the magnetic phase. Following the determination of these parameters, we perform ionic position optimizations to obtain the ground states for both the ferromagnetic and antiferromagnetic phases, ultimately identifying the more stable magnetic configuration for each monolayer.  

Subsequent calculations utilize the PBEsol functional to assess potential improvements in electronic properties compared to existing experimental and theoretical studies of the monolayers. To account for the strong correlation among \(d\) electrons in each monolayer, Hubbard \(U\) corrections are applied using both PBE and PBEsol functionals, which modify the magnetic moments of the transition metals and the potential band gaps, aligning closely with experimental and theoretical values. Full optimization calculations are conducted to evaluate the preservation of symmetry in the monolayers, facilitated by FINDSYM utility \supercite{stokes2005findsym}.  

Phonon calculations are performed using a \(3 \times 3 \times 1\) supercell for the stable magnetic phase of CrGeTe$_{3}$, and a \(4 \times 4 \times 1\) supercell for both MnGeTe$_{3}$ and FeGeTe$_{3}$. The vibrational properties of these structures are studied via the “atomic force from finite displacements” method implemented in the Phonopy package \supercite{phonopy-phono3py-JPCM,phonopy-phono3py-JPSJ}. The functional that most accurately describes each monolayer, based on previous analysis, is utilized for these calculations.  

Finally, we investigate ferromagnetic random alloys at three different concentrations: \(x = 0.25\), \(x = 0.50\), and \(x = 0.75\). These random alloys take the forms Cr\(_{x}\)GeMn\(_{1-x}\)Te\(_{3}\), Cr\(_{x}\)GeFe\(_{1-x}\)Te\(_{3}\), and Fe\(_{x}\)GeMn\(_{1-x}\)Te\(_{3}\), each containing 24 formula units (f.u.). The alloys are generated using the mcsqs code \supercite{atat5}, which implements a Monte Carlo algorithm to find the best special quasi-random structure. This code is part of the Alloy-Theoretic Automated Toolkit (ATAT) \supercite{atat1,atat2,atat3,atat4,atat5,atat6,atat7,atat8,atat9,atat10,atat11,atat12,atat13,atat14,atat15,atat16}. Subsequently, VASP is employed for electronic optimizations, utilizing only the PBE functional. 
\section{Flow of VASP}

\begin{itemize}
	\item VASP selects the appropriate pseudopotential method for the calculation, using the Projector-Augmented Wave (PAW) method. This method ensures a precise treatment of core electrons while maintaining computational efficiency. Additionally, VASP employs an exchange-correlation functional as specified by the user.
	\item The initial electronic density is computed.
	\item VASP iteratively constructs the Kohn-Sham Hamiltonian in a self-consistent (SC) manner.
	\item The electronic density is updated at each iteration, and the effective potential is recalculated until the self-consistent cycle converges to a predefined tolerance level. This tolerance, typically set by the user, is recommended to be $1 \times 10^{-6}$ eV, as suggested in the VASP documentation \supercite{EDIFF}. See Fig. \ref{fig:cycle_scf}.
\end{itemize}

\begin{figure}[H]
	\includegraphics[width=14cm]{Figures/cycle_scf.jpg}
	\centering
	\caption{Flowchart adapted from the works of Gross\supercite{Gross2003} and Barhoumi\supercite{barhoumi2021}, illustrating the self-consistent field (SCF) cycle in density functional theory (DFT) calculations, as implemented in VASP. The cycle starts with an initial guess for the electron density $n^{0}(\mathbf{r})$, followed by the construction of the effective potential $v^{j}_{\text{eff}}(\mathbf{r})$, incorporating the Hartree potential, exchange-correlation functional, and pseudopotentials. The Kohn-Sham equations are then solved to obtain updated wavefunctions, which are used to recalculate the electron density. This iterative process continues until the difference between successive effective potentials falls below a predefined tolerance, typically $1 \times 10^{-6}$ eV \supercite{EDIFF}. The cycle ensures convergence to the ground-state electronic structure \supercite{Kohn1965,blochl1994}.}
	\label{fig:cycle_scf}
\end{figure}
\section{VASP Input and Output Files}

This section provides a detailed explanation of the required input files for VASP calculations and the output files generated during the process.

\subsection{Input Files}
To perform a VASP calculation, four essential input files are required: INCAR, POSCAR, KPOINTS, and POTCAR.

\subsubsection{INCAR}
The INCAR file contains the parameters that govern various aspects of the calculation in VASP. Each parameter influences different stages of the simulation, such as electronic structure optimization, Density of States (DOS) calculations, ionic relaxation, and magnetization settings.

\begin{figure}[H]  
	\centering  
	\begin{threeparttable}  
		\caption{Example of an INCAR file used for optimizing ionic positions while keeping the cell size and shape fixed. This file provides the total energy of the system. By applying it to different unit cell sizes, the Two-Dimensional Equation of State \ref{2D.eqs} can be used to determine the optimal lattice parameters. For further details, refer to the VASP manual.}
		\label{fig:fig3.1}  
		\resizebox{\textwidth}{!}{  
			\begin{tabular}{>{\columncolor{blue!10}}l>{\columncolor{blue!10}}l>{\columncolor{blue!10}}l>{\columncolor{blue!10}}l>{\columncolor{blue!10}}l>{\columncolor{blue!10}}l>{\columncolor{blue!10}}l>{\columncolor{blue!10}}l>{\columncolor{blue!10}}l>{\columncolor{blue!10}}l>{\columncolor{blue!10}}l>{\columncolor{blue!10}}l>{\columncolor{blue!10}}l>{\columncolor{blue!10}}l>{\columncolor{blue!10}}l>{\columncolor{blue!10}}l>{\columncolor{blue!10}}l>{\columncolor{blue!10}}l>{\columncolor{blue!10}}l}  
				\hline   
				\multicolumn{3}{c}{\cellcolor{blue!10} \textbf{GENERAL}}& & & & & & & & & & & & & & & &\\   
				\textbf{SYSTEM} & \textbf{= XGeTe$_3$} & \textit{\# System name defined by the user}& & & & & & & & & & & & & & & & \\   
				\textbf{PREC}   & \textbf{= Accurate} & & & & & & & & & & & & & & & & &\\
				\multicolumn{3}{c}{\cellcolor{blue!10} \textbf{ELECTRONIC OPTIMIZATION}} & & & & & & & & & & & & & & & &\\   
				\textbf{ENCUT}  & \textbf{= 500}  & \textit{\# Cutoff energy (in eV)}& & & & & & & & & & & & & & & & \\   
				\textbf{LREAL}  & \textbf{= Auto} & & & & & & & & & & & & & & & & &\\
				\textbf{ISMEAR} & \textbf{= 0}    & \textit{\# -5 for insulators, 1 or 2 for metals} & & & & & & & & & & & & & & & &\\   
				\textbf{ALGO}   & \textbf{= Normal} & & & & & & & & & & & & & & & & &\\
				\multicolumn{3}{c}{\cellcolor{blue!10} \textbf{DOS CALCULATION}}& & & & & & & & & & & & & & & & \\   
				\textbf{LORBIT} & \textbf{= 11}  & \textit{\# Necessary for DOS calculations}& & & & & & & & & & & & & & & &\\   
				\textbf{NEDOS}  & \textbf{= 4000} & \textit{\# Number of DOS points}& & & & & & & & & & & & & & & &\\   
				\multicolumn{3}{c}{\cellcolor{blue!10} \textbf{IONIC RELAXATION}} & & & & & & & & & & & & & & & &\\   
				\textbf{ISIF}   & \textbf{= 2}   & \textit{\# Optimize ionic positions only \tnote{a}}& & & & & & & & & & & & & & & & \\   
				\textbf{IBRION} & \textbf{= 2}   & \textit{\# Conjugate gradient algorithm}& & & & & & & & & & & & & & & &\\   
				\textbf{EDIFF}  & \textbf{= 1E-06} & \textit{\# Energy convergence criterion} & & & & & & & & & & & & & & & &\\   
				\textbf{NELM}   & \textbf{= 100} & \textit{\# Maximum number of SCF iterations} & & & & & & & & & & & & & & & &\\   
				\textbf{NSW}    & \textbf{= 100} & \textit{\# Maximum number of ionic steps}& & & & & & & & & & & & & & & &\\   
				\multicolumn{3}{c}{\cellcolor{blue!10} \textbf{MAGNETIZATION}}& & & & & & & & & & & & & & & &\\   
				\textbf{ISPIN}  & \textbf{= 2}   & \textit{\# Spin-polarized calculation}& & & & & & & & & & & & & & & &\\   
				\textbf{MAGMOM} & \textbf{= 2*3 2*0 6*0} & \textit{\# Initial magnetic moments \tnote{b}} & & & & & & & & & & & & & & & &\\   
				\hline  
			\end{tabular}  
		}  
		\begin{tablenotes}  
			\footnotesize  
\begin{minipage}{\textwidth}
	\item[a] By default ISIF$=3$ optimizes ionic positions, cell shape, and cell volume. For monolayers, cell size relaxation is typically restricted to the x and y directions. To achieve this, modifications are made to the default 'vasp\_std' script to perform the desired two-dimensional cell size optimization.
	\item[b] Magnetic moments are assigned according to the atom order in the POSCAR and POTCAR files. In this example: 2 atoms of X $\times$ $\mu_B$(X), 2 atoms of Ge $\times$ $\mu_B$(Ge), and 6 atoms of Te $\times$ $\mu_B$(Te).
\end{minipage}
		\end{tablenotes}  
	\end{threeparttable}  
\end{figure}

Proper definition of the tags in the INCAR file is essential. Incorrect or unnecessary tag usage can lead to unrealistic results. Therefore, it is crucial to be meticulous and always refer to the official VASP manual for guidance.
 

\subsubsection{POSCAR}
The POSCAR file serves as a comprehensive blueprint for defining the atomic structure of materials and molecules in computational simulations. It consists of several key sections, each containing specific information essential for accurate simulations. The first line is often a comment for human readability, providing a brief description of the system without affecting the calculation. Following this, the lattice vectors are provided, which define the shape and size of the unit cell. These vectors can be specified either in Cartesian or fractional coordinates, as illustrated in Figure \ref{fig:fig3.2}.
\begin{figure}[H]
\resizebox{\textwidth}{!}{
	\begin{tabular}{>{\columncolor{blue!10}}c>{\columncolor{blue!10}}c>{\columncolor{blue!10}}c>{\columncolor{blue!10}}l>{\columncolor{blue!10}}l>{\columncolor{blue!10}}l>{\columncolor{blue!10}}l>{\columncolor{blue!10}}l>{\columncolor{blue!10}}l>{\columncolor{blue!10}}l>{\columncolor{blue!10}}l>{\columncolor{blue!10}}l>{\columncolor{blue!10}}l>{\columncolor{blue!10}}l>{\columncolor{blue!10}}>{\columncolor{blue!10}}l>{\columncolor{blue!10}}l>{\columncolor{blue!10}}l>{\columncolor{blue!10}}l>{\columncolor{blue!10}}l>{\columncolor{blue!10}}l>{\columncolor{blue!10}}l>{\columncolor{blue!10}}l>{\columncolor{blue!10}}l>{\columncolor{blue!10}}l>{\columncolor{blue!10}}l}\hline
		\multicolumn{3}{l}{\cellcolor{blue!10} \textbf{ X Ge Te}} & & & & & & & & & & & & & & & & & & & & & &\\
		\multicolumn{3}{l}{\cellcolor{blue!10}1.0}& & & & & & & & & & & & & & & & & & & & & &\\
		6.9068061733 & 0.0000000000 & 0.0000000000 & & & & & & & & & & & & & & & & & & & & & &\\
		-3.4534030866 & 5.9814696051 & 0.0000000000& & & & & & & & & & & &  & & & & & & & & & &\\
		0.0000000000 & 0.0000000000 & 21.8184108734 & & & & & & & & & & & & & & & & & & & & & &\\
		\multicolumn{3}{l}{\cellcolor{blue!10} \textbf{ X Ge Te }}& & & & & & & & & & & && & & & & & & & & & \\
		\multicolumn{3}{l}{\cellcolor{blue!10} 2   2   6}& & & & & & & & & & & & & & & & & & & & & &\\
		\multicolumn{3}{l}{\cellcolor{blue!10}Direct} & & & & & & & & & & & & & & & & & & & & & &\\
		0.666666687 & 0.333333343 & 0.500000000 & & & & & & & & & & & & & & & & & & & & & &\\
		0.333333343 & 0.666666687 & 0.500000000 & & & & & & & & & & & & & & & & & & & & & &\\
		0.000000000 & 0.000000000 & 0.444593012& & & & & & & & & & & & & & & & & & & & & &\\
		0.000642670 & 0.373393325 & 0.421062350& & & & & & & & & & & & & & & & & & & & & &\\
		0.999357343 & 0.626606705 & 0.578937650 & & & & & & & & & & & & & & & & & & & & & &\\
		0.626606705 & 0.627303362 & 0.421062350& & & & & & & & & & & & & & & & & & & & & &\\
		0.373393325 & 0.372696668 & 0.578937650& & & & & & & & & & & & & & & & & & & & & &\\
		0.372696668 & 0.999357343 & 0.421062350 & & & & & & & & & & & & & & & & & & & & & &\\
		0.627303362 & 0.000642670 & 0.578937650& & & & & & & & & & & & & & & & & & & & & &\\ \hline
	\end{tabular}
 }
	\centering
	\caption{Unit cell structure in fractional coordinates for the XGeTe$_3$ (X = Cr, Mn, Fe) monolayer, containing 10 atoms: 2 X atoms, 2 Ge atoms, and 6 Te atoms.}
	\label{fig:fig3.2}
	\label{poscar}
\end{figure}

A scaling factor, typically provided after the lattice vectors, allows for resizing the simulation cell without altering the atomic coordinates. The file then specifies the atomic species present in the system and their corresponding quantities, whether they are standard elements or user-defined labels.

In essence, the POSCAR file provides comprehensive input for computational simulations, ensuring an accurate representation of the atomic structure within the system. For further details, refer to the official VASP manual\supercite{POSCAR}.
\subsubsection{KPOINTS}
The KPOINTS file provides essential instructions for defining the sampling of the Brillouin zone, which is a fundamental region in reciprocal space related to the periodic structure of a crystal. The choice of k-points significantly affects the accuracy of the calculations by determining how the wavefunctions are sampled in this region.
\begin{figure}[H]
\resizebox{\textwidth}{!}{
	\begin{tabular}{>{\columncolor{blue!10}}c>{\columncolor{blue!10}}l>{\columncolor{blue!10}}l>{\columncolor{blue!10}}l>{\columncolor{blue!10}}l>{\columncolor{blue!10}}l>{\columncolor{blue!10}}l>{\columncolor{blue!10}}l>{\columncolor{blue!10}}l>{\columncolor{blue!10}}l>{\columncolor{blue!10}}l>{\columncolor{blue!10}}l>{\columncolor{blue!10}}l>{\columncolor{blue!10}}l>{\columncolor{blue!10}}l>{\columncolor{blue!10}}l>{\columncolor{blue!10}}l>{\columncolor{blue!10}}l>{\columncolor{blue!10}}l>{\columncolor{blue!10}}l>{\columncolor{blue!10}}l>{\columncolor{blue!10}}l>{\columncolor{blue!10}}l>{\columncolor{blue!10}}l>{\columncolor{blue!10}}l>{\columncolor{blue!10}}l>{\columncolor{blue!10}}l>{\columncolor{blue!10}}l>{\columncolor{blue!10}}l>{\columncolor{blue!10}}l>{\columncolor{blue!10}}l>{\columncolor{blue!10}}l>{\columncolor{blue!10}}l>{\columncolor{blue!10}}l>{\columncolor{blue!10}}l>{\columncolor{blue!10}}l>{\columncolor{blue!10}}l} \hline
		\multicolumn{1}{l}{\cellcolor{blue!10} \textbf{k-points}} & & & & & & & & & & & & & & & & & & & & & & & & & & & & & & & & & & & &\\ 
		\multicolumn{1}{l}{\cellcolor{blue!10}0}& & & & & & & & & & & & & & & & & & & & & & & & & & & & & & & & & & & &\\
		\multicolumn{1}{l}{\cellcolor{blue!10} \textbf{Gamma}}& & & & & & & & & & & & & & & & & & & & & & & & & & & & & & & & & & & &\\ 
		6 6 1& & & & & & & & & & & & & & & & & & & & & & & & & & & & & & & & & & & & \\
		0 0 0& & & & & & & & & & & & & & & & & & & & & & & & & & & & & & & & & & & & \\  \hline
	\end{tabular}
 }
	\centering
	\caption{K-point grid centered at the Gamma point in the Brillouin zone. The dimensions "6 6 1" define the grid in the x, y, and z directions, respectively. The value "1" in the z direction indicates a 2D system, such as a monolayer.}
	\label{fig:fig3.3}
	\label{kpoints}
\end{figure}
Different schemes can be employed to select k-points. One commonly used method is the Monkhorst-Pack scheme, which ensures a uniform distribution of points across the Brillouin zone. The Brillouin zone is constructed from the reciprocal lattice and represents a Wigner-Seitz cell, characterized by paths that pass through high-symmetry points. For the systems under study, this path is $\Gamma$-$M$-$K$-$\Gamma$. See Fig. \ref{fig:brillouin_zone} for a top view of the Brillouin zone and the corresponding high-symmetry points.

\begin{figure}[H]
	\centering
	\includegraphics[width=10cm]{Figures/brillouin_zone.png}
	\caption{Top view of the Brillouin zone of XGeTe$_{3}$ monolayers, showing the path through the high-symmetry points that define its Wigner-Seitz unit cell.}
	\label{fig:brillouin_zone}
\end{figure}

Alternatively, Gamma-centered grids can be used, as shown in Figure \ref{fig:fig3.3}, where the k-points are centered at the Gamma point of the Brillouin zone. The automatic generation of k-points is also an option depending on the complexity and symmetry of the system under study.

Each k-point can be assigned a weight, indicating its relative contribution to the total energy and other physical properties calculated. Proper weighting of k-points is crucial for achieving accurate and efficient results. For additional details, please refer to the official VASP manual\supercite{KPOINTS}.
\subsubsection{POTCAR}
The POTCAR file acts as a specialized instruction manual for computers during computational simulations. It aids the computer in comprehending the behavior of atoms within materials. Within this manual, default settings dictate the computer's calculations. These settings encompass various aspects, including the selection of exchange-correlation functionals, and the utilization of simplified atom models such as PAW.

These details are meticulously chosen based on both theoretical principles and experimental observations to ensure that the computer's predictions closely align with real-world phenomena. Recognizing the significance of the POTCAR file cannot be overstated. It serves as the bedrock for the computer's operations. By adhering to the appropriate settings and rules outlined in the POTCAR file, the computer can generate highly accurate predictions regarding the behavior of materials.

\subsection{Output files}
Upon completing ab initio calculations, several output files provide crucial information about the simulation results. These files offer insights into the electronic structure, energy convergence, atomic positions, and other important parameters. Here are brief descriptions of some of the most important output files:

\subsubsection{OUTCAR}
The \texttt{OUTCAR} file is a comprehensive output log that contains detailed information from the electronic structure calculations. This includes convergence criteria, energy values, forces acting on atoms, magnetic moments, and electronic band structure data. It serves as a valuable source for analyzing the simulation results in depth.

\subsubsection{CONTCAR}
The \texttt{CONTCAR} file contains the final atomic coordinates after the relaxation or optimization process. It is essential for visualizing the relaxed atomic structure and can be used as an input for subsequent calculations, ensuring continuity between simulations.

\subsubsection{DOSCAR}
The \texttt{DOSCAR} file provides information about the density of states (DOS), which represents the distribution of electronic states across a specified energy range. This file is essential for analyzing the electronic properties of a system and understanding its behavior under various conditions. One significant feature of this file is that it allows the calculation of the number of valence electrons, \( N_v \), which characterize the system under study. The number of valence electrons can be computed using the following equation:

\begin{equation}
	N_v = \int_{-\infty}^{\varepsilon_{F}} n(\varepsilon) \, d\varepsilon 
\end{equation}

Here, \( n(\varepsilon) \) is the density of states as a function of energy \( \varepsilon \), and \( \varepsilon_{F} \) is the Fermi energy. This integration is performed over the energy states up to the Fermi level.

\subsubsection{PROCAR}
The \texttt{PROCAR} file contains data about the projected electronic band structure, revealing the contributions of each atomic orbital to the electronic states. This information is indispensable for analyzing the electronic structure and the orbital character of the material.

\subsubsection{OSZICAR}
The \texttt{OSZICAR} file records the convergence behavior of both the electronic and ionic iterations during the calculation. It provides details on the energy convergence and the progress of the optimization, which is important for assessing the reliability and accuracy of the simulation results.

\section{Phonon calculations using Phonopy software}

The process of phonon calculation begins with a full relaxation of the atomic structure, where all degrees of freedom are optimized until the forces on each atom converge to a stable configuration. This optimization step generates a \texttt{CONTCAR} file, which contains the final equilibrium positions of atoms. Subsequently, supercells are constructed, where atoms are displaced to simulate the vibrational modes of the material. The displacements are calculated using Phonopy with the following command:

\begin{figure}[H]
	\begin{tabular}{>{\columncolor{black}}p{\linewidth}}
		\texttt{\textcolor{green!70!black}{username:\textcolor{blue}{$\sim$}}\textcolor{white}{\$} \textcolor{white}{phonopy -d --dim="2 2 1" --magmom="3 3 0 0 0 0 0 0 0 0"}} \\ \hline
	\end{tabular}
	\centering
	\caption{Creation of displacements for a supercell: The command \texttt{phonopy -d --dim} generates the displaced structures required for force calculations. The \texttt{--magmom} tag is essential when dealing with magnetic atoms.}
	\label{fig:fig3.4}
	\label{displacements}
\end{figure}

The underlying theory behind lattice vibrations is essential for understanding the thermodynamic, mechanical, and vibrational properties of crystalline solids. In a crystal, each atom oscillates around its equilibrium position, giving rise to vibrational modes, or phonons, which can either be acoustic or optical. As detailed by \textit{Kittel}\supercite{kittel2021} and \textit{Ashcroft}\supercite{ashcroft1976}, these modes are crucial for explaining key properties such as heat capacity, thermal conductivity, and electron-phonon interactions.

The nuclear positions in a crystal lattice are described as:
\begin{equation}
	\mathbf{R}_I(t) = \mathbf{R}_l + \mathbf{u}_I(t),
\end{equation}
where \( \mathbf{u}_I(t) \) represents the small displacement from equilibrium, and \( \mathbf{R}_l \) refers to the lattice points in the crystal. These displacements are often modeled within the harmonic approximation. According to Newton's second law, the equation of motion for these displacements is:
\begin{equation}
	M_I \ddot{\mathbf{u}}_I(t) = -\frac{\partial U}{\partial \mathbf{u}_I(t)},
\end{equation}
where \( M_I \) denotes the mass of the nucleus, and \( U \) represents the total potential energy of the system. In equilibrium, \( \frac{\partial U}{\partial \mathbf{u}_I} = 0 \), indicating that the system is stable. The potential energy \( U \) is expanded as a Taylor series around the equilibrium positions, and its second-order terms describe the harmonic interactions between neighboring atoms through the Born-von Karman force constants:
\begin{equation}
	K_{ls\alpha, l's'\beta} = \frac{\partial^2 U}{\partial (R_{l\alpha} + \tau_{s\alpha}) \partial (R_{l'\beta} + \tau_{s'\beta})}.
\end{equation}

From here, the equation of motion for the small displacements \( u_{I\alpha}(t) \) is reduced to:
\begin{equation}
	M_I \ddot{u}_{I\alpha}(t) = - \sum_{J\beta} \Phi_{I\alpha, J\beta} u_{J\beta}(t),
\end{equation}
where \( \Phi_{I\alpha, J\beta} \) are the force constants. These force constants are central to constructing the dynamical matrix, defined as:
\begin{equation}\label{eq:3.3}
	D_{s\alpha, s'\beta}(\mathbf{q}) = \frac{1}{(M_s M_{s'})^{1/2}} \sum_{l} e^{i\mathbf{q} \cdot \mathbf{R}_l} e^{i\mathbf{q} \cdot (\tau_{s'} - \tau_s)} K_{0s\alpha, ls'\beta},
\end{equation}
where \( D_{s\alpha, s'\beta}(\mathbf{q}) \) represents the dynamical matrix with the phonon wavevector \( \mathbf{q} \). Diagonalizing this matrix yields eigenvalues, corresponding to the square of the phonon frequencies \( \omega^2 \), and eigenvectors that represent the vibrational eigenmodes.  The eigenvalues distinguish between acoustic phonons, which are characterized by long-wavelength, in-phase oscillations of atoms, and optical phonons, where neighboring atoms oscillate out of phase. For a more in-depth analysis, see Chapter 7 of \textit{Giustino}'s book\supercite{Giustino2014}.

After generating the displaced supercells using Phonopy, VASP is employed to compute the forces acting on atoms in each supercell, as is indicated in Eq. \ref{eq:3.3}. This is achieved through single-point calculations, where the \texttt{INCAR} file is configured with specific tags tailored to this task, see Fig. \ref{incar_pho}. These calculations provide the force data necessary for constructing the force constants matrix \( \Phi \), which is crucial for determining the phonon frequencies and related vibrational properties.

\begin{figure}[H]
\resizebox{\textwidth}{!}{
\begin{tabular}{>{\columncolor{blue!10}}l>{\columncolor{blue!10}}l>{\columncolor{blue!10}}l>{\columncolor{blue!10}}l>{\columncolor{blue!10}}l>{\columncolor{blue!10}}l>{\columncolor{blue!10}}l>{\columncolor{blue!10}}l>{\columncolor{blue!10}}l>{\columncolor{blue!10}}l>{\columncolor{blue!10}}l>{\columncolor{blue!10}}l>{\columncolor{blue!10}}l>{\columncolor{blue!10}}l>{\columncolor{blue!10}}l>{\columncolor{blue!10}}l>{\columncolor{blue!10}}l>{\columncolor{blue!10}}l>{\columncolor{blue!10}}l>{\columncolor{blue!10}}l>{\columncolor{blue!10}}l>{\columncolor{blue!10}}l>{\columncolor{blue!10}}l>{\columncolor{blue!10}}l>{\columncolor{blue!10}}l>{\columncolor{blue!10}}l>{\columncolor{blue!10}}l}\hline
	\multicolumn{3}{c}{\cellcolor{blue!10} \textbf{GENERAL}} & & & & & & & & & & & & & & & & & & & & & & & &\\ 
	\textbf{SYSTEM} & = phonon calculation & & & & & & & & & & & & & & & & & & & & & & & & &\\ 
	\textbf{PREC}   & = Normal  &   & & & & & & & & & & & & & & & & & & & & & & & &\\ 
	\textbf{ISTART}   & = 0   &  & & & & & & & & & & & & & & & & & & & & & & & &\\ 
	\textbf{ICHARG}   & = 2  &   & & & & & & & & & & & & & & & &  & & & & & & & & \\ 
	\multicolumn{3}{c}{\cellcolor{blue!10} }& & & & & & & & & & & & & & & & & & & & & & & & \\ 
	\textbf{ENCUT}   & = 500   & & & & & & & & & & & & & & & & &  & & & & & & & &\\
	\textbf{SIGMA}   & = 0.05  & & & & & & & & & & & & & & & & & & & & & & & & &\\ 
	\textbf{ALGO}   & = Normal  & & & & & & & & & & & & & & & & & & & & & & & & &\\ 
	\textbf{LREAL}   & = .FALSE. &  & & & & & & & & & & & & & & & &  & & & & & & & &\\ 
	\textbf{EDIFF}   & = 1E-07   & & & & & & & & & & & & & & & & &  & & & & & & & &\\ 
	\textbf{NELM}   & = 100   &  & & & & & & & & & & & & & & & & & & & & & & & &\\ 
	\multicolumn{3}{c}{\cellcolor{blue!10} \textbf{OPTIMIZATION}}& & & & & & & & & & & & & & & &  & & & & & & & &\\ 
	\textbf{NSW} & = 0  & & & & & & & & & & & & & & & & &  & & & & & & & & \\ 
	\textbf{ISIF} & = 2 & & & & & & & & & & & & & & & & &  & & & & & & & & \\ 
	\multicolumn{3}{c}{\cellcolor{blue!10}} & & & & & & & & & & & & & & & &  & & & & & & & &\\ 
	\textbf{POTM} & = 0.5    & & & & & & & & & & & & & & & & &  & & & & & & & &\\
	\textbf{ISMEAR} & = 0    &  & & & & & & & & & & & & & & & & & & & & & & & &\\ 
	\textbf{IBRION} & = -1    & & & & & & & & & & & & & & & & & & & & & & & & &\\ 
	\textbf{LORBIT} & = 11   &  & & & & & & & & & & & & & & & & & & & & & & & &\\ 
	\multicolumn{3}{c}{\cellcolor{blue!10}} & & & & & & & & & & & & & & & & & & & & & & & &\\ 
	\textbf{ISPIN}   & = 2   &   & & & & & & & & & & & & & & & & & & & & & & & & \\ 
	\textbf{MAGMOM}  & = 8*3 8*0 24*0 &   & & & & & & & & & & & & & & & &  & & & & & & & &\\ 
	\multicolumn{3}{c}{\cellcolor{blue!10}} & & & & & & & & & & & & & & & & & & & & & & & &\\
	\textbf{LWAVE}   & = .FALSE. & & & & & & & & & & & & & & & & &  & & & & & & & &\\ 
	\textbf{ADDGRID}    & = .TRUE. & & & & & & & & & & & & & & & & &  & & & & & & & &\\ \hline
\end{tabular}
}
\centering
\caption{Configuration of the \texttt{INCAR} file for single-point calculations. Note the use of the \texttt{IBRION=-1} tag for this purpose. The magnetic order is specified by the \texttt{MAGMOM} tag, which must be consistent with the displacements generated in Fig. \ref{displacements}.}
\label{fig:fig3.5}
\label{incar_pho}
\end{figure}

To compute the forces from the VASP output file \texttt{vasprun.xml}, the following Phonopy command is used:

\begin{figure}[H]  
	\begin{tabular}{>{\columncolor{black}}p{\linewidth}}  
		\texttt{\textcolor{green!70!black}{username:\textcolor{blue}{$\sim$}}\textcolor{white}{\$} \textcolor{white}{phonopy -f pho\_POSCAR\{01..10\}/vasprun.xml}} \\ \hline  
	\end{tabular}  
	\centering  
	\caption{Command to compute force sets for each displaced supercell.}  
	\label{fig:fig3.6}  
	\label{forces}  
\end{figure}  

The computed forces are then used to derive key phonon properties, such as phonon dispersion curves (phonon bands) and the phonon density of states (DOS). These properties are calculated using Phonopy with the following commands:

\begin{figure}[H]  
	\resizebox{\textwidth}{!}{  
		\begin{minipage}[l]{\textwidth}  
			\subcaption{}  
			\centering  
			\begin{tabular}{>{\columncolor{blue!10}}l>{\columncolor{blue!10}}l>{\columncolor{blue!10}}l>{\columncolor{blue!10}}l>{\columncolor{blue!10}}l}\hline   
				\multicolumn{5}{l}{\cellcolor{black!100} \texttt{\textcolor{green!70!black}{username:\textcolor{blue}{$\sim$}}\textcolor{white}{\$} \textcolor{white}{phonopy -p -s band.conf}}} \\   
				\multicolumn{5}{c}{\cellcolor{white!10} } \\ \hline   
				\textbf{ATOM-NAME} & = Cr Ge Te & & &\\   
				\textbf{EIGENVECTORS}   & = .TRUE.  &  & &  \\   
				\textbf{DIM}   & = 2 2 1   &  & & \\   
				\textbf{BAND}   & = 0 0 0 & 0.5 0 0   &  0.333 0.33 0 & 0 0 0 \\   
				\textbf{BAND-POINTS}   & = 1001   & & &  \\   
				\textbf{BAND-LABELS}   & = $\Gamma$  M K $\Gamma$ &  & &\\  \hline  
			\end{tabular}  
			
			% Note below the table  
			\footnotesize \textit{Note: The high symmetry points ($\Gamma$, M, K) are indicative of the zone boundaries in the Brillouin zone and were obstained using the python module SeeK-path}.  
		\end{minipage}  
	}  
	\hfill  
	\resizebox{\textwidth}{!}{  
		\begin{minipage}[l]{\textwidth}  
			\subcaption{}  
			\centering  
			\begin{tabular}{>{\columncolor{blue!10}}l>{\columncolor{blue!10}}l>{\columncolor{blue!10}}l}\hline   
				\multicolumn{3}{l}{\cellcolor{black!100} \texttt{\textcolor{green!70!black}{username:\textcolor{blue}{$\sim$}}\textcolor{white}{\$} \textcolor{white}{phonopy -p -s dos.conf}}} \\   
				\multicolumn{3}{c}{\cellcolor{white!10} } \\ \hline  
				\textbf{DIM} & = 3 3 1 & \\   
				\textbf{MP}   & = 51 51 51   &   \\   
				\textbf{DOS}   & = .TRUE.   &  \\   
				\textbf{GAMMA-CENTER}   & = .TRUE.&  \\   
				\textbf{FPITCH}   & = 0.02  &  \\ \hline  
			\end{tabular}  
		\end{minipage}  
	}  
	\centering  
	\caption{Commands used to obtain (a) phonon band structure based on high-symmetry points provided by the SeeK-path tool \supercite{Hinuma2017,Togo2018}, and (b) phonon density of states (DOS).}  
	\label{pho_dos_bands}  
\end{figure}



\section{Implementation for generate random magnetic alloys: ATAT}

The theoretical background of this software package was discussed in Section \ref{sqs}. Here, we outline the input files required to generate quasi-random structures. Specifically, we employ the Monte Carlo approach (mcsqs method), which requires two input files: `rndstr.in` and `sqscell.out`.

The first file, `rndstr.in` (see Fig. \ref{fig:3.8} $\mathbf{a}$), is similar to the POSCAR file. It contains the lattice vectors and atomic positions of the system in its unit cell form, but also allows for partial occupations of atomic positions. The second file, `sqscell.out` (see Fig. \ref{fig:3.8} $\mathbf{b}$), specifies the supercell required to generate a Special Quasirandom Structure (SQS) that matches the desired compositions of the random alloy. 

For example, we cannot use a $3\times3\times1$ supercell for a desired composition of $x=0.25$ in a disordered (Cr,Mn) sublattice of the unit cell $Cr_{x}Mn_{1-x}GeTe_{3}$, since the smallest step in that supercell would be $1/3^2 = 1/9$. Therefore, the allowed concentrations for this supercell would be: 
$x=0, 1/9, 2/9, 1/3, 4/9, 5/9, 2/3, 7/9, 8/9, 1.$ \\

\begin{figure}[H]
	% First figure on the left

	\hspace{-0.5cm}
	% Second figure in the center
	\begin{minipage}[t]{0.1\linewidth}
				\renewcommand{\arraystretch}{1.70}
				\begin{flushleft}
		\vspace{-0.65cm}
		\captionsetup{position=top}
		\subcaptionbox{}{
		\resizebox{9.5cm}{!}{
\begin{tabular}{>{\columncolor{blue!10}}l>{\columncolor{blue!10}}c>{\columncolor{blue!10}}c>{\columncolor{blue!10}}c>{\columncolor{blue!10}}l>{\columncolor{blue!10}}c} \hline 
	&6.9068061733 & 0.0000000000 & 0.0000000000  & &\\
	&-3.4534030866 & 5.9814696051 & 0.0000000000 & &\\
	&0.0000000000 & 0.0000000000 & 21.8184108734 & &\\
   {\cellcolor{blue!10} 1   0  0} &  & & & &\\
   {\cellcolor{blue!10} 0   1   0}  & & & & &\\
   {\cellcolor{blue!10} 0  0   1}   & & & & &\\
	&0.666666687 & 0.333333343 & 0.500000000  & Cr=0.25, & Mn=0.75 \\
	&0.333333343 & 0.666666687 & 0.500000000  & Cr=0.25, & Mn=0.75\\
	&0.000000000 & 0.000000000 & 0.444593012 & Ge &\\
	&0.000642670 & 0.373393325 & 0.421062350    & Ge&\\
	&0.999357343 & 0.626606705 & 0.578937650   &  Te &\\
	&0.626606705 & 0.627303362 & 0.421062350    & Te&\\
	&0.373393325 & 0.372696668 & 0.578937650    & Te&\\
	&0.372696668 & 0.999357343 & 0.421062350    & Te&\\
	&0.627303362 & 0.000642670 & 0.578937650    & Te&\\ \hline
    \end{tabular}
    }
    }
	\label{fig:3.8a}
	\end{flushleft}
	\end{minipage}
	\hspace{8cm}
	% Third figure on the right
	\begin{minipage}[t]{0.2\linewidth}
		\vspace{-0.6cm}
		\begin{flushright}
			\captionsetup{position=top}
		    \subcaptionbox{}{
			\resizebox{2.8cm}{!}{
\begin{tabular}{>{\columncolor{blue!10}}c>{\columncolor{blue!10}}c>{\columncolor{blue!10}}c}\hline
	 1  & &  \\
     \multicolumn{3}{c}{\cellcolor{blue!10}} \\ 
	4& 0 &0  \\
	0 & 3& 0 \\
    0 &0 & 1 \\ \hline
\end{tabular}
		}
	}
	\label{fig:3.8b}
		\end{flushright}
	\end{minipage}
	\caption{$\mathbf{(a)}$ Format of the `rndstr.in` file, where the desired composition can be set in a unit cell lattice system. For instance, $x=0.25$ for Cr atoms and $1-x=0.75$ for Mn atoms in the unit cell of $Cr_{1-x}Mn_{x}GeTe_{3}$ (with two formula units per unit cell), randomly distributed. $\mathbf{(b)}$ The structure of the `sqscell.out` file, which generates the desired SQS (in this case, a $4\times3\times1$ supercell). This system consists of 12 unit cells (i.e., 24 formula units) with a total of 120 atoms.}
	\label{fig:3.8}
\end{figure}

With the `rndstr.in` file, we can construct the cluster expansions required by the `corrdump` code to obtain the cluster expansion using the following command:

\begin{figure}[H]
	\begin{threeparttable}
	\begin{tabular}{>{\columncolor{black}}p{\linewidth}}
		\texttt{\textcolor{green!70!black}{username:$\sim$}\textcolor{white}{\$} \textcolor{white}{corrdump -l=rndstr.in -ro -noe -nop -clus -2=13.83 -3=10.56 -4=6.91}} \\ \hline
	\end{tabular}
		\centering
	\caption{Command used to perform cluster expansions for the $Cr_{x}Mn_{1-x}GeTe_{3}$ unit cell system at $x=0.25$, where Cr and Mn atoms are randomly distributed.}
	\label{fig:fig3.9}
	\begin{tablenotes}
		\footnotesize 
		\item Note:  We specify the cluster expansion for figures of two vertices up to the sixth nearest neighbor (NN), figures of three vertices up to the third NN, and figures of four vertices up to the second NN. The NN values are obtained by running the same command with a larger cutoff distance for two-vertex figures (e.g., 20 \AA), ensuring that the relevant NN are included. These figures are displayed in the 'clusters.out' file.
	\end{tablenotes}
\end{threeparttable}
\end{figure}

The command generates two output files: 'clusters.out', which lists all the identified clusters, and 'sym.out', which determines the space group of the system described in the 'rndstr.in' file. These files are then used for the Monte Carlo simulation, performed using the following command:

\begin{figure}[H]
	\begin{threeparttable}
		\begin{tabular}{>{\columncolor{black}}p{\linewidth}}
			\texttt{\textcolor{green!70!black}{username:\textcolor{blue}{$\sim$}}\textcolor{white}{\$} \textcolor{white}{mcsqs -rc}} \\ \hline
		\end{tabular}
		\centering
		\caption{Command to generate the special quasirandom structure (SQS) using the Monte Carlo algorithm.}
		\label{fig:4.0}
		\begin{tablenotes}
			\footnotesize 
			\item Note: The '-rc' flag allows for specifying the desired supercell, which is defined in the 'sqscell.out' file.  
		\end{tablenotes}
	\end{threeparttable}
\end{figure}


Finally, the 'bestsqs.out' file is generated. This file follows a format similar to that of the 'POSCAR' file and can be directly used for subsequent calculations. Due to the large supercell generated by the mcsqs method, several electronic optimization steps in VASP will be necessary within a relaxation cycle to ensure proper convergence.

