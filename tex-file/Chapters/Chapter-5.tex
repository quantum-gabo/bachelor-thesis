\chapter{\texorpdfstring{Conclusions $\&$ Outlook}{Conclusions \& Outlook}} % Main chapter title

\label{Chapter5} % For referencing the chapter elsewhere, use \ref{Chapter5}
\lhead{Chapter 5. \emph{Conclusions $\&$ Outlook}} % This 

In this work, we successfully integrate first-principles calculations with machine learning techniques to develop a robust and efficient machine learning force field (MLFF) tailored for calcium silicate hydrates (C-S-H). The MLFF was constructed using an on-the-fly training approach within \emph{ab initio} molecular dynamics (AIMD) simulations, enabling the accurate modeling of the atomic and mechanical properties of C-S-H. Starting with geometric relaxation of the C-S-H structure using VASP and the PBEsol exchange-correlation functional, we employed a systematic workflow to train, evaluate, and refine the MLFF. This refined force field was then used to compute key thermodynamic properties, including the equation of state (EOS) and mechanical parameters such as the bulk modulus.

Comparisons across different MLFF variants and with available experimental data demonstrated the critical importance of the refinement step in enhancing model accuracy and transferability.
Our final MLFF showed good agreement with experimental mechanical properties, confirming its capability to capture the complex atomic interactions within C-S-H. Nevertheless, there is still room for improvement and future work that could further enhance the performance and applicability of the proposed MLFFs. A notable direction for future work involves validating the bulk parameters obtained with the MLFFs against DFT calculations, which would provide a better assessment of their accuracy. Additionally, some future improvements may include, but are not limited to, incorporating Van der Waals (vdW) corrections to account for long-range interactions and employing more advanced exchange-correlation functionals. However, the feasibility of these improvements will depend on the available computational resources.

Finally, machine learning-based approaches, as presented in this work, hold great promise for advancing materials research, particularly in the context of large and complex systems, where first-principles methods can be computationally prohibitive. In this regard, machine learning-based concrete research could significantly accelerate the development of more durable and sustainable concrete, addressing the pressing environmental challenges associated with its production and use.
