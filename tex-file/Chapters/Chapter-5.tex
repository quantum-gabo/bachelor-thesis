\chapter{\texorpdfstring{Conclusions $\&$ Outlook}{Conclusions \& Outlook}} % Main chapter title

\label{Chapter5} % For referencing the chapter elsewhere, use \ref{Chapter5}
\lhead{Chapter 5. \emph{Conclusions $\&$ Outlook}} % This 

In this work, first-principles calculations and machine learning techniques were successfully integrated to construct a robust and efficient machine learning force field (MLFF) tailored for calcium silicate hydrates (C--S--H). The model was developed through an on-the-fly training scheme within \emph{ab initio} molecular dynamics (AIMD) simulations, enabling accurate representation of the atomic-scale interactions governing the structure and mechanics of C--S--H. Beginning with the relaxation of the C--S--H unit cell using VASP and the PBEsol exchange-correlation functional, a systematic workflow was established to train, validate, and refine the MLFF.

The results of molecular dynamics and relaxation simulations confirm that the developed MLFF faithfully reproduces the energetics and mechanical response of C--S--H, achieving close agreement with first-principles data in terms of total energy, atomic forces, and stress tensors. Moreover, the force field demonstrates excellent predictive capability for macroscopic mechanical properties—such as the bulk modulus---yielding values consistent with experimental findings. Finally, transferability tests across a temperature range of 200--400~K revealed that the MLFF reliably captures the thermal behaviour of C--S--H, offering a computationally efficient tool for exploring its structural and thermodynamic properties without requiring further training.


Nonetheless, there remain opportunities for further improvement of the methodology presented herein. Notable directions for future work include validating the bulk parameters obtained with the MLFFs against DFT calculations to provide a better assessment of their accuracy. Additional improvements could involve the use of more advanced exchange–correlation functionals along with van der Waals dispersion corrections to enhance the accuracy of the force field, evaluating the performance of the MLFF using a larger supercell, and expanding the training dataset to include a wider range of C--S--H compositions to improve the model's generalisability.

Finally, machine learning-based approaches, as presented in this work, hold great promise for advancing materials research, particularly in the context of large and complex disordered systems like C--S--H, where first-principles methods can be computationally prohibitive. In this regard, machine learning-based concrete research could significantly accelerate the development of more durable and sustainable concrete, addressing the pressing environmental challenges associated with its production and use.



