% Chapter 5
\chapter{\texorpdfstring{Conclusions $\&$ Outlook}{Conclusions \& Outlook}} % Main chapter title

\label{Chapter5} % For referencing the chapter elsewhere, use \ref{Chapter5}

\lhead{Chapter 5. \emph{Conclusions $\&$ Outlook}} % This is for the header on each page - perhaps a shortened title

This work provides a comprehensive analysis of the magnetic and electronic properties of XGeTe$_3$ (X = Cr, Mn, Fe) monolayers and their random alloys, including Cr$_{1-x}$GeMn$_{x}$Te$_3$, Cr$_{1-x}$GeFe$_{x}$Te$_3$, and Fe$_{1-x}$GeMn$_{x}$Te$_3$, using density functional theory (DFT) with PBE and PBESol functionals, alongside Hubbard $U$ corrections.

For the CrGeTe$_3$ monolayer, we confirmed robust ferromagnetic (FM) ordering, driven by strong exchange interactions. The PBE+U functional, with $U = 3.0$ eV, offered a balanced trade-off between computational efficiency and accuracy. The calculated band gap aligned well with experimental data. Phonon stability analysis confirmed the dynamical stability of the system, supporting its potential for practical applications. Future studies may benefit from using more advanced functionals for deeper exploration of its electronic and magnetic properties.

The MnGeTe$_3$ monolayer exhibited half-metallic (HM) behavior, making it suitable for spintronic applications. The PBE+U functional, with $U = 2.5$ eV, effectively captured both the magnetic moments and ground state energies. Phonon analysis confirmed its dynamical stability, positioning MnGeTe$_3$ as a promising material for further refinement of the Hubbard $U$ parameter or exploration of alternative functionals.

The FeGeTe$_3$ monolayer displayed distinct deviations from the symmetrical behavior observed in CrGeTe$_3$ and MnGeTe$_3$. These deviations were attributed to its antiferromagnetic (AFM) ground state and the hybridization between Fe $d$ states and Te $p$ states. Phonon analysis revealed negative frequencies, suggesting potential dynamical instabilities that need to be addressed to fully understand and optimize the electronic properties of FeGeTe$_3$.

For the random alloys, substitution of Cr, Mn, and Fe atoms in XGeTe$_3$ supercells led to notable changes in magnetic moments, electronic structures, and alloy stability. In Cr$_{1-x}$GeMn$_{x}$Te$_3$ at $x = 0.50$, magnetic moment disorder was observed, where one Mn atom exhibited a negative magnetic moment due to local atomic environments and magnetic frustration linked to variations in Mn-Te bond lengths. This suggests the presence of complex magnetic ground states, which could be beneficial for spintronic applications. Negative formation energies across different concentrations confirmed the alloy's global thermodynamic stability, while positive mixing energies indicated structural complexity that could give rise to non-ferromagnetic phases.

In Cr$_{1-x}$GeFe$_{x}$Te$_3$, at $x = 0.75$, strong hybridization between Fe and Cr $d$ states and Te $p$ states was observed, indicating the tunability of electronic transport properties through Fe doping. This positions the alloy as a promising candidate for applications in magnetic sensors and thermoelectric devices. The consistent magnetic moments and stable formation energies across concentrations further affirm its technological viability.

In the Fe$_{1-x}$GeMn$_{x}$Te$_3$ system, increasing Mn concentrations (up to $x = 0.75$) led to structural tension and strong hybridization between Mn and Fe $d$ states with Te. This suggests the alloy’s potential for applications in magnetic tunneling junctions and other devices requiring tailored electronic and magnetic properties. However, despite the negative formation energies confirming thermodynamic stability, the structural tensions at high Mn concentrations must be addressed to fully harness the alloy's technological potential.

In conclusion, this investigation underscores the significant technological potential of XGeTe$_3$ monolayers and their random alloys for applications in spintronics, magnetic memory, and thermoelectric devices. Precise control over magnetic frustration, hybridization, and alloy stability is essential for the successful deployment of these materials, paving the way for future research in this rapidly advancing field.