% Chapter 1

\chapter{Introduction} % Main chapter title

\label{Chapter1} % For referencing the chapter elsewhere, use \ref{Chapter1} 

\lhead{Chapter 1. \emph{Introduction}} % This is for the header on each page - perhaps a shortened title

Understanding how materials function at their fundamental levels is crucial for driving technological progress. Modelling and simulations, based on computational algorithms and supported by scientific theories \supercite{Xiao2007,Lee2011}, play a vital role in this process. They provide detailed insights into the behavior of materials at the atomic and molecular levels, enabling the design of new materials tailored to specific applications, such as storage devices \supercite{Anurag2021}, bio-materials, etc. This interdisciplinary field, commonly known as materials science, aims to accurately predict the properties of novel materials, ensuring alignment with experimental observations. The consistent validation of these predictions through experimentation, along with the pursuit of dependable and precise forecasts, is paramount for the success of this emerging discipline \supercite{Giustino2014}.

In this context, the exploration of two-dimensional (2D) materials, exemplified by the groundbreaking discovery of graphene in 2004 \supercite{Novoselov2004}, has drawn significant attention to this class of structures. These materials, characterized by a thickness of up to about a nanometer \supercite{Paul2017}, present a unique spectrum of properties and diverse applications.

However, a significant challenge arises from the fact that many 2D materials are not thermodynamically stable and are expensive to synthesize due to the poor control of synthesized layers \supercite{Choi2022}. This challenge has prompted researchers to turn to theoretical simulations, using density functional theory (DFT), to guide the quest for new stable configurations.  Examples like boron nitride \supercite{Cassabois2016} or transition metal dichalcogenides (TMDs) \supercite{Splendiani2010} highlight the rich possibilities inherent in 2D materials, ranging from tunable bandgaps\supercite{Ramasubramaniam2011} to intriguing electronic properties such as quantum confinement effects\supercite{Ding2019}. Another interesting class of materials that includes transition metal compounds are the transition-metal trichalcogenides (TMTCs) with the chemical formula ABX$_3$ \supercite{Sivadas2015}. These materials display intriguing magnetic properties that can be utilized to understand the little-studied field of magnetism in 2D systems, which began with the experimental synthesis of magnetic CrGeTe$_3$\supercite{Gong2017} and CrI$_3$\supercite{Huang2017} monolayers in 2017. Furthermore, they offer insights into how magnetism can be confined to only one plane and controlled by the thickness of the layers, owing to spin-orbit coupling or magneto-crystalline dipole-dipole interactions \supercite{Gibertini2019}.

These intriguing phenomena, intimately linked to the behavior of d-spin electronic orbitals within transition metal atoms, can be addressed by GGA$+$U\supercite{Anisimov1991}, a variant of the Generalized Gradient Approximation (GGA), in combination with the PBESol functional \supercite{Perdew2008}.  Such an approach,  which has not been thoroughly studied for XGeTe3 monolayers can be allow the accurate description of on-site electron-electron interactions and provides improved predictions of structural and vibrational properties, respectively.
%----------------------------------------------------------------------------------------

\section{Problem Statement}

\label{section.problem_statement}

We intend to utilize the approach described above to conduct ab initio simulations using density functional theory on TMTCs materials like XGeTe3 monolayers, where X represents a transition metal such as Cr (Chromium), Mn (Manganese), or Fe (Iron). This computational study aims to shed light on their potential applications and will be compared with theoretical and experimental studies on MnGeTe$_3$\supercite{Chittari2020,Song2023,Hao2021} 2D systems. Finally, magnetic alloys will be examined among these three monolayers.

%----------------------------------------------------------------------------------------

\section{General ans Specific Objectives}

\label{section.general_specific_objectives}
The primary objective of these computational studies is to employ advanced density functional theory methods for detailed investigations on monolayers of XGeTe$_3$, where X represents chromium (Cr), manganese (Mn), or iron (Fe). Thus, the following steps are undertaken:

\begin{itemize}
    \item Explanation of the DFT basis and exchange-correlated functionals such as PBE, PBESol, and their enhancement with DFT$+$U formalism.
    \item Description of the formalism utilized to study XGeTe$_3$ monolayers, executed through calculations on the VASP package.
    \item Examination of the electronic, magnetic, and vibrational properties of XGeTe$_3$ monolayers using PBE and PBESol functionals.
    \item Performing Hubbard U corrections to accurately describe XGeTe$_3$ monolayers.
    \item Exploration of the formation of magnetic alloys by combining XGeTe$_3$ monolayers.
    \item Drawing conclusions to highlight key findings and implications.
\end{itemize}

